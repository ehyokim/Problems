\documentclass[12pt]{article}%
\usepackage{amsfonts}
\usepackage{amsmath}
\usepackage{fancyhdr}
\usepackage{comment}
\usepackage{braket}
\usepackage{qcircuit}
\usepackage[a4paper, top=2.5cm, bottom=2.5cm, left=2.2cm, right=2.2cm]%
{geometry}
\usepackage{times}
\usepackage{amsmath}
\usepackage{changepage}
\usepackage{amssymb}
\usepackage{graphicx}%
\setcounter{MaxMatrixCols}{30}
\newtheorem{theorem}{Theorem}
\newtheorem{acknowledgement}[theorem]{Acknowledgement}
\newtheorem{algorithm}[theorem]{Algorithm}
\newtheorem{axiom}{Axiom}
\newtheorem{case}[theorem]{Case}
\newtheorem{claim}[theorem]{Claim}
\newtheorem{conclusion}[theorem]{Conclusion}
\newtheorem{condition}[theorem]{Condition}
\newtheorem{conjecture}[theorem]{Conjecture}
\newtheorem{corollary}[theorem]{Corollary}
\newtheorem{criterion}[theorem]{Criterion}
\newtheorem{definition}[theorem]{Definition}
\newtheorem{example}[theorem]{Example}
\newtheorem{exercise}[theorem]{Exercise}
\newtheorem{lemma}[theorem]{Lemma}
\newtheorem{notation}[theorem]{Notation}
\newtheorem{problem}[theorem]{Problem}
\newtheorem{proposition}[theorem]{Proposition}
\newtheorem{remark}[theorem]{Remark}
\newtheorem{solution}[theorem]{Solution}
\newtheorem{summary}[theorem]{Summary}
\newenvironment{proof}[1][Proof]{\textbf{#1.} }{\ \rule{0.5em}{0.5em}}

\newcommand{\Q}{\mathbb{Q}}
\newcommand{\R}{\mathbb{R}}
\newcommand{\C}{\mathbb{C}}
\newcommand{\Z}{\mathbb{Z}}
\newcommand{\N}{\mathbb{N}}

\begin{document}

\title{Problems in Folland's Real Analysis}
\author{Edward Kim}
\date{\today}
\maketitle


\section*{Chapter 5 (Functional Analysis)}

\subsection{Folland 5.16}

\subsection{Folland 5.21}
\begin{proof}
Let $\alpha: X^* \times Y^* \rightarrow (X \times Y)^*$ be the map $(f,g) \mapsto f\circ \pi_1 + g \circ \pi_2$ where $\pi_1,\pi_2$ are the canonical projection maps onto $X,Y$ respectively. This naturally can be evaluated as:
$\alpha(f,g)(x,y) = f(x) + g(y)$. First, note that $\alpha \in L(X^* \times Y^*, (X \times Y)^*)$. To see this, note the following expressions hold for any $\|(f,g)\| \neq 0$ and $\| x,y \| \neq 0$:
$$\| \alpha(f,g)(x,y) \| = | f(x) + g(y) | \leq \| f\|\| x\| \cdot \|g \| \| y\|  \leq \| f\| \cdot \max\{ \|x\|, \|y\| \} + \| g\| \cdot \max\{ \|x\|, \|y\| \}$$
$$ = (\| (x,y) \|) \cdot (\| f\| + \| g \|) $$
Therefore,
$$ \| \alpha(f,g) \| = \sup_{\|(x,y)\| \neq 0 }\frac{\| \alpha(f,g)(x,y) \|}{\| (x,y) \|}\leq \| (f,g)\| $$
To show that $\alpha$ is surjective, let $r \in (X \times Y)^*$:
$$ r((x,y)) = r((x,0) + (0,y)) = r((x,0)) + r((0,y))$$ Defining $f: X \rightarrow \R$ as the map $ x \mapsto r((x,0))$ and $g: Y \rightarrow \R$ as $y \mapsto r((0,y))$, surjectivity immediately follows. Now, suppose there exists $(f_1,g_1), (f_2,g_2) \in X^* \times Y^*$ such that $\alpha(f_1,g_1) = \alpha(f_2,g_2)$. In other words, for all $(x,y) \in X \times Y$:
$$ f_1(x) + g_1(y) = f_2(x) + g_2(y) $$
Taking $y = 0$ yields that $f_1 = f_2$ and taking $x = 0$ yields $g_1 = g_2$. proving injectivity. Hence, $\alpha$ is a bijection. To show that $\alpha$ is an isometry, it suffices to prove that $\| \alpha(f,g) \| \geq \|(f,g)\|$.
\end{proof}

\subsection{Folland 5.24}
\begin{enumerate}
\item
\begin{proof}
Let $g \in \widehat{X^0} \cap \widehat{X^*}$.
\end{proof}
\item
\begin{proof}
Suppose $X$ is reflexive. By definition, $\hat{X} = X^{**}$. By part one above, $\widehat{X^0}$ becomes:
$$ \widehat{X^0} = \{ F \in X^{***} \mid F |_{X^{**}} = F = 0 \}$$ Hence,
$\widehat{X^0} = \{ 0 \}$. Invoking our result from part one, we conclude that $\widehat{X^*} = X^{***}$ which proves that $X^*$ is reflexive. Conversely, if $X^*$ is reflexive, then, once again, $\widehat{X^0} = \{ 0 \}$. For the sake of contradiction, assume that
$X$ is not reflexive. Then there must exist $g \in X^{**} \backslash \hat{X}$. Since $\hat{X}$ is a subspace of $X^{**}$, define a linear map $f: \hat{X} \rightarrow \R$ such that $f = 0$ on $\hat{X}$. Extend $f$ to the subspace generated by $g$ such that $f(g) \neq 0$ by $f(x + \lambda g) = \lambda \cdot t$ for some $t \geq \inf_{y \in \hat{X}} \{ \|g - y\|\} \neq 0$. Let $p(x) = \| x\|$ be the sublinear function that dominates $f$ on the subspace $\hat{X} + \mathbb{F}g$ where $\mathbb{F}$ can be taken over $\R$ or $\C$ (See Theorem 5.8). By Hahn-Banach, we conclude that there exists a linear functional $F: X^{**} \rightarrow \R$ such that $F|_{\hat{X}} = 0$ and $F(g) \neq 0$. This directly contradicts our assumption above that $\widehat{X^0} = \{ 0 \}$. Hence, it immediately follows that $X^*$ must be reflexive.
\end{proof}

\begin{remark}
This question exhibits that reflexivity "bubbles" its way up the dual spaces. In other words, if $X^{**}$ can be described as applying points in $X$ to our functionals  in $X^*$, then, if we view points in $X^*$ as points in a vector space, $X^{***}$ can be described as applying points $X^*$ to our functionals in $X^{**}$. We can see the general picture as follows: Let $X$ be a Banach space (We need this to ensure that $\bar{\hat{X}} = \hat{X}$). Denote the $n^{th}$ dual space as $X^{*(n)}$. Then each dual space has a natural image two levels up:
$$ X^{*(n)} \mapsto \widehat{X^{*(n)}} \subseteq X^{*(n+2)}$$
By part two, if $X$ is reflexive, then so is every dual space above it. (More on this later)
\end{remark}

\end{enumerate}


\end{document}
