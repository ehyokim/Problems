\documentclass[Lang.tex]{subfiles}

\begin{document}

\section*{Lang Chapter 2 (Rings)}

\subsection{Problem 10}

Let $D$ be an integer greater than or equal to 1, and let $R$ be the set consisting of elements in the form $a + b\sqrt{-D}$ where $a,b \in \mathbb{Z}$.

\begin{enumerate}
	\item 
		\begin{proof}
			We begin by showing that $R$ is indeed a ring under the usual addition and multiplication operations over $\mathbb{C}$. We can see that indeed $r + s = (a_1 + b_1\sqrt{-D}) + (a_2 + b_2\sqrt{-D} = (a_1 + a_2) + (b_1 + b_2)\sqrt{-D}$ for $r,s \in R$. Similarly, $rs = a_1a_2 - b_1b_2D + (a_1b_2 + a_2b_1)\sqrt{-D} \in R$.
			\end{proof}
			\item
				\begin{proof}
						 We observe that the above observations make $R$ into a subring of $\mathbb{C}$. Thus, there exists an embedding $\phi: R \rightarrow \mathbb{C}$ which is a ring homomorphism. Let $\star: \mathbb{C} \rightarrow \mathbb{C}$ denote the complex conjugation map. As $\star$ is an automorphism, we can compose the maps $\phi^{-1} \circ \star \circ \phi: R \rightarrow R$ to yield an automorphism for $R$.s  
				\end{proof}
				
\end{enumerate}

\subsection*{Problem 11}
$\bullet$ Define the ring of trigonometric functions as the polynomial ring $\mathbb{R}[\sin(x),\cos(x)]$. 

\begin{enumerate}
	\item
		\begin{proof}
			We shall prove that the elements $f(x)$ of the trigonometric polynomial ring $R$ above can be expressed in the following form:
			$$ f(x) = a_0 + \sum_{m=1}^n a_m \sin(mx) + b_m \cos(mx)  $$
			where $a_m,b_m,a_0 \in \mathbb{R}$.  
			
			Every $f \in \mathbb{R}[\sin(x),\cos(x)]$ can be reduced to the form above by associativity and commutivity of the addition and multiplication operations in the field $\mathbb{R}$ and subsequently point-wise multiplication of $\sin(x),\cos(x) \in C(\mathbb{R})$. We invoke the following identities to reduce products of $\sin^m(x),\cos^m(x)$ to products of $\sin(mx), \cos(mx)$:
			$$ \cos^2(x) = \frac{1 + \cos(2x)}{2} $$
			$$ \sin^2(x) = \frac{1 - \cos(2x)}{2} $$
			
			We note that by the first and second identities, we can reduce terms of the forms $\cos^{2m}(x),\sin^{2n}(x)$ to the desired forms $(\frac{1 + \cos(2x)}{2})^m, (\frac{1 - \cos(2x)}{2})^m$ respectively. Futhermore, we have the product-to-sum identities:
			
			$$ \sin(x)\sin(y) = \frac{1}{2}[\cos(x - y) - \cos(x+y))] $$
			$$ \cos(x)\cos(y) = \frac{1}{2}[\cos(x - y) + \cos(x+y))] $$
			
			It is routine verification to see that the combination of the above identities with associativity yields the desired forms for terms $\sin^m(x),\cos^m(x)$. We handle terms of mixed powers of $\sin(x), \cos(x)$ by the similar product-to-sum identity:
			$$ \sin(x)\cos(y) = \frac{1}{2}[\sin(x - y) + \sin(x-y))] $$
			Once again, we inductively reduce the powers of $\sin(x),\cos(x)$ to the forms above then invoke the mixed product-to-sum identity. Combining these reductions, we can inductively reduce the elements of the trigonometric polynomial ring to the desired form, thus concluding the proof.
			
			\end{proof}
			\item 
		
			    	\begin{proof}
			    		We now prove that $\text{deg}_{tr}(fg) = \text{deg}_{tr}(f) + \text{deg}_{tr}(g)$. Without loss of generality, let $f = a_0 + ... + a_r \cos(rx)$ and $g = b _0 + ... + b_s \cos(sx)$. By multiplying $f,g$, we yield $fg a_0b_0 + .. + a_rb_s\cos(rx)\cos(sx)$. By the product-to-sum identities above, we can reduce this product to the sum
			    		
			    		$$ a_rb_s\cos(rx)\cos(sx) = \frac{a_rb_s}{2}[\cos((r-s)x) + \cos((r+s)x)] $$
			    		
			    		Since $a_r,b_r \neq 0$, $a_rb_r \neq 0$, and $\cos(r+s)x$ is the term with maximum degree by definition above. The equality immidiately follows and we see that $R$ cannot have any zero divisors since the sum of two positive degrees can never be zero. 
					\end{proof}
				\end{enumerate}
			
\subsection*{Problem 12}			
 $\bullet$  Let $P$ be the set of positive integers. Define the ring $R$ as the set of the functions defined on the set $P$ with values in a commutative ring $K$ with the sum defined to be the pointwise addition of functions and the convolution product to be dictated by the formula
 $$ (f \star g)(m)  = \sum_{xy = m} f(x)g(y) $$ 
\begin{enumerate}
	\item 
	\begin{proof}
		We first prove that $R$ is a commutative ring with the unit of $R$ defined to be the function $\delta$ which takes $\delta(1) = 1_K$ and $\delta(x) = 0_K$ for $x \neq 1$. To see that $R$ is commutative we note that by the commutativity of $K$, $$ \sum_{xy = m}f(x)g(y) = \sum_{yx = m} g(y)f(x) $$ By commutativity of $P$, taking the sum over all factors of $m$ yields  $$\sum_{yx = m} g(y)f(x) = \sum_{xy = m} g(x)f(y) = (g\star f)(m)$$ Let $\delta$ be the function as defined above. For $f \in R$.
		$$ (f \star \delta)(m) = \sum_{xy = m} f(x)\delta(y) =  f(m)\delta(1)  = f(m) $$ 
		
		for $m \in P$. As the convolution product agrees with $f$ for all elements in $P$, $\delta$ serves as our unit $1_R$.
	\end{proof}
	\item 
		\begin{proof}
			Suppose we have multiplicative functions $f,g \in R$. Let $m,n \in P$ be relatively prime postive integers. Then:
			$$ (f \star g)(mn) = \sum_{xy = mn} f(x)g(y) $$ We note that $x,y$ can be decomposed into $x = x_1y_1$ and $y = x_2y_2$ where $x_1x_2  = m$ and $y_1y_2 = n$. Since $m,n$ are relatively prime, it follows that $x_1,y_1$ and $x_2,y_2$ are also relatively prime. Hence, we can decompose the product as such
			$$ \sum_{xy = mn} f(x)g(y) = \sum_{xy = mn} f(x_1)f(y_1)g(x_2)g(y_2) = \sum_{xy = mn} f(x_1)g(x_2)f(y_1)g(y_2) =$$
			$$ [\sum_{x_1y_1 = m} f(x_1)g(y_1)][\sum_{y_1y_2 = n} f(y_1)g(y_2)] = (f \star g)(m)(f \star g)(n)$$ Thus, we see that the product is also multiplicative. We see in particular that the set of multiplicative functions endowed with the addition and multiplication operations of $R$ forms a subring of $R$.
		\end{proof}
	\item
		\begin{proof}
			Let $\mu$ be the Mobius function defined as follows: $\mu(1) = 1,  \mu(p_1...p_r) = (-1)^r$ for $p_1,...,p_r$ distinct primes, and $\mu(x) = 0$ if $p^2 | x$ for some prime $p$. We first prove that $\mu$ is multiplicative. Indeed, let $m,n$ be relatively prime postive integers. We can product $mn$ as a product of distinct prime powers. $\mu(mn) = 0$ if either $m,n$ contains a non-trivial prime power. Thus, it suffices to take the case where $mn$ can be written as a product of single prime powers. Let $m = p_1...p_r$ and $n = p_{r+1}...p_{r+s}$ where all primes are distinct, then $\mu(mn) = (-1)^{r+s} = (-1)^r(-1)^s = \mu(m)\mu(n)$. Hence, $\mu$ is multiplicative. 
			
			Now consider the convolution product $ (\mu * \phi_1) $ where $\phi_1$ is the constant function taking values to $1$. We can expand the product as follows: Let $m = p_1^{a_1}...p_r^{a_r}$
			
			$$ (\mu \star \phi_1)(p_1^{a_1}...p_r^{a_r}) = \sum_{i = 1}^r {r \choose i} (-1)^i $$  If $m = 1$, then the product is $1$ trivially. If $m \neq 1$, then the sum vanishes (proof?). Thus, $(\mu \star \phi_1) = \delta$. 
		\end{proof}
	\end{enumerate}
	
	\subsection*{Problem 15 (Dedekind Rings)}
		$\bullet$ Let $\mathfrak{o}$ be a subring of field $K$ such that every element of $K$ can be expressed as a quotient of elements of $\mathfrak{o}$ \newline 
		$\bullet$ Define a fractional ideal $\mathfrak{a}$ as a non-zero additive subgroup of $K$ such that $\mathfrak{o}\mathfrak{a} \subseteq \mathfrak{a}$. Since $\mathfrak{o}$ contains the unit element, $\mathfrak{o}\mathfrak{a} = \mathfrak{a}$. Also, we require that there exists a $c \in \mathfrak{o}$ such that $c\mathfrak{a} \subset \mathfrak{o}$. The first property can be interpreted as the "closure of the numerators" i.e for $a \backslash b \in \mathfrak{a} $, $oa \backslash b \in \mathfrak{a}$ for all $o \in \mathfrak{o}$. The second property bounds the denominator to $\mathfrak{o}$. \newline
		$\bullet$ A Dedekind ring  $\mathfrak{o}$ is ring as above such that its fractional ideals form a group under multiplication with the identity element as $\mathfrak{o}$. 
		
		\begin{enumerate}
			\item We claim that every fractional ideal $\mathfrak{a}$ is finitely generated. Since the set of fractional ideasl under multiplication forms a group, there exists ideal $\mathfrak{b}$ such that $\mathfrak{a}\mathfrak{b} = \mathfrak{o}$. Thus there exists $a_1,...,a_n \in \mathfrak{a}$ and $b_1,...,b_n \in \mathfrak{b}$ such that:
			$$ 1 = \sum a_ib_i$$. Hence, for any $a \in \mathfrak{a}$, 
			$$ a = \sum (ab_i)a_i $$. This shows that $\mathfrak{a}$ is finitely generated by the elements $a_1,...,a_n$.
			\item 
		\end{enumerate}

\end{document}