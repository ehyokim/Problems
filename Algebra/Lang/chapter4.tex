\documentclass[Lang.tex]{subfiles}


\begin{document}

\section*{Lang Chapter 4 (Polynomials)}

\subsection*{Problem 1}
	$\bullet$ Let $k$ be a field and $f(x) \in k[x]$ be a non-zero polynomial. 
	\begin{proof}
		Suppose that the ideal $(f(x))$ is maximal, then the ideal must also be prime. Now suppose that the ideal is prime, then if $g,h \in k[x]$ such that $gh \in ()f(x)))$ and either $g \in (f(x))$ or $h \in (f(x))$. Note that this implies irreducibility as if $f$ were reducible, it's factors $f_1f_2 \in (f(x))$, but neither $f_1,f_2$ would be contained in the ideal as both are not units. Finally, suppose that $f(x)$ is irreducible. Take $q \in k[x]$  such that $(q(x)) + (f(x))$ is proper ideal containing $(f(x))$. Since $k[x]$ is a principal ideal domain, there exists $z \in k[x]$ such that $(z(x))$ generates the ideal. Hence, $z|f$. However, by  the irreducibility of $f$, $z = uf$ where $u \in k$.  Hence, $(q(x)) \subseteq (f(x))$, a contradiction. This shows that the three criteria are indeed equivalent. 
	\end{proof}

	This shows that in the polynomial ring $k[x]$, the prime ideals are exactly the maximal ideals which are once again exactly characterized by the ideals generated by the irreducible polynomials. In the perspective of the Zariski topology the closed set $V(E)$ for $E \subseteq k[x]$ is generated by the common irreducible factors of each polynomial in $E$.

\section{Problem}


\end{document}