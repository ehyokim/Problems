\documentclass[12pt]{article}%
\usepackage{amsfonts}
\usepackage{amsmath}
\usepackage[a4paper, top=2.5cm, bottom=2.5cm, left=2.2cm, right=2.2cm]%
{geometry}
\usepackage{times}
\usepackage{complexity}
\usepackage{graphicx}
\usepackage{physics}
\usepackage{amsmath}
\usepackage{amssymb}
\newenvironment{proof}[1][Proof]{\textbf{#1.} }{\ \rule{0.5em}{0.5em}}

\begin{document}

\title{Homological Quantum Codes and Hyperbolic 4-Manifolds}
\author{Edward Kim}
\date{\today}
\maketitle

We propose the final project to expound on the current literature of classes of Quantum Low Density Parity Check Codes \cite{breuckmann2021quantum} based on Homological Invariants and some of their relations to the Geometry of Hyperbolic 4-manifolds. The treated topics will generally range in those found in High-dimensional Codes from 4-manifolds \cite{guth2014quantum, hastings2016quantum}, Systolicity \cite{fetaya2011homological, freedman2002z2}, Tesselation Codes over Hyperbolic 4-manifolds \cite{londe2017golden, breuckmann2017homological}, Homological Product Codes \cite{bravyi2014homological, delfosse2021union}, or the recent Fibre Bundle Code developed By Hastings, Haah, and O'Donnell \cite{hastings2021fiber}. The current tentative plan is to mostly focus on the Guth-Lubotzky Construction of LDPC Codes over Arithmetic Hyperbolic 4-manifolds and some connections to Systolic Geometry. 

\nocite{*}
\bibliographystyle{abbrv}
\bibliography{propbib}


\end{document}
