\documentclass[12pt]{article}%
\usepackage{amsfonts}
\usepackage{amsmath}
\usepackage[a4paper, top=2.5cm, bottom=2.5cm, left=2.2cm, right=2.2cm]%
{geometry}
\usepackage{times}
\usepackage{complexity}
\usepackage{graphicx}
\usepackage{physics}
\usepackage{amsmath}
\usepackage{amssymb}
\newenvironment{proof}[1][Proof]{\textbf{#1.} }{\ \rule{0.5em}{0.5em}}

\begin{document}

\title{ECE 621 HW 7}
\author{Edward Kim}
\date{\today}
\maketitle

\begin{enumerate}
	\item 
		\begin{description}
			\item[Surface-13:] $[[9,1,3]]$
			\item[Surface-17:] $[[9,1,3]]$ 	
			\item[Surface-25:] $[[13,1,3]]$ 	
		\end{description}

	\item Consider Figure 1(b) depicting the Surface-17 code. For the sake of demonstration, we restrict our attention to a $Z$-error on syndrome measurement ancilla qubit $13$. In the original scheme depicted in Figure 5(a), such a $Z$ error would induce $Z$-errors on qubits $6, 7$, note that the $X$-type stabilizer measured by qubit $16$ would beunaffected by the $Z$-errors on qubits $6,7$. Hence, after a round of the surface code syndrome measurement procedure, only the syndrome measurement performed by qubit $14$ would yield a $-1$ outcome. However, the decoding algorithm would take this measurement and output a $Z$ operation on either qubit $5$ or $8$. Either of these operations would lead to a chain of $Z$-errors equivalent to a logial $Z$ operation up to stabilizers.

	On the other hand, if we reorder the CNOT gates according to Figure 5(b), we see that qubits $3,6$ have $Z$ errors. This would entail that the $X$-type stabilizers measured by ancilla qubits $11,16$. The decoder algorithm would perform a $Z$ operation on qubit $6$ and either qubits $3$ or $6$. This would correctly fix the error detected by the syndrome measurement of qubit $13$.

\item Recall that as a CSS code, the Steane $[[7,1,3]$ code corrects $X$-type and $Z$-type errors independently by applying a parity check matrix to the possibly corrupted logical state. This amounts to a sequence of CNOT gates with $7$ qubits acting as control qubits and $3$ ancilla qubits (initialized to $\ket{0}$) for both $X$-type and $Z$-type stabilizers (see Nielsen, Chuang Exercise 10.26). For both of these independent correction procedures, each ancilla will be set to the parity after applying then corresponding row of the parity check matrix.  Since each stabilizer measurement corresponding to a row of the parity check matrix will be performed using seperate ancilla qubits, it does not matter which order we perform the CNOT gates. 
%This follows from the observation

\item 
	
\end{enumerate}

\end{document}
