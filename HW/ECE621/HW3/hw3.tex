
\documentclass[12pt]{article}%
\usepackage{amsfonts}
\usepackage{amsmath}
\usepackage[a4paper, top=2.5cm, bottom=2.5cm, left=2.2cm, right=2.2cm]%
{geometry}
\usepackage{times}
\usepackage{physics}
\usepackage{amsmath}
\usepackage{amssymb}
\newenvironment{proof}[1][Proof]{\textbf{#1.} }{\ \rule{0.5em}{0.5em}}

\begin{document}

\title{ECE 621 HW 3}
\author{Edward Kim}
\date{\today}
\maketitle

\section*{Problem 1}
\begin{enumerate}
	\item The state $\ket{\psi} = \ket{0} \otimes \ket{+} \otimes \ket{+i}$
		is stablized by the subgroup generated by $\langle ZII, IXI, IIY \rangle$ as each qubit state is the eigenvector associated to the $+1$ eigenvalue or trivially stabilized by the identity.
		\begin{align*}
			\label{eq:}
			Z \ket{0}  &= \ket{0} \\
			X \ket{+} &= \ket{+} \\
			Y \ket{+i} &= \ket{+i}
		\end{align*}
	\item An alternative generator representation of the stabilizer subgroup can be:
		\begin{equation*}
			\langle ZXI, IXI, IXY \rangle
		\end{equation*}
	\item We can transform the stabilizer subgroup above into the desired one through the following series of Clifford operations.
		\begin{align*}
			\langle ZII, IXI, IIY \rangle & \xrightarrow{H^{\otimes 2} \otimes I_3} \\
			\langle XII, IZI, IIY \rangle & \xrightarrow{CNOT_{1,2} \otimes I_3} \\	
			\langle XXI, ZZI, IIY \rangle & \xrightarrow{I^{\otimes 2} \otimes H} \\
			\langle XXI, ZZI, -IIY \rangle & \xrightarrow{I^{\otimes 2} \otimes S} \\
			\langle XXI, ZZI, IIX \rangle & \xrightarrow{I^{\otimes 2} \otimes H} \\
			\langle XXI, ZZI, IIZ \rangle & \xrightarrow{I_1 \otimes CNOT_{2,3}} \\
			\langle XXX, ZZI, IZZ \rangle
		\end{align*}

	\item Iterating through all positive one eigenspaces of the generators, we see that the stabilized state of the generators derived above will be:
		\begin{equation*}
			% \ket{\psi'} = \ket{+} \otimes \ket{+} \otimes \ket{+}
			\ket{\psi'} = \frac{1}{\sqrt{4}} \left[ \ket{000} + \ket{110} + \ket{011} + \ket{111} \right]
		\end{equation*}
\end{enumerate}

\section*{Problem 2}
Recall the stabilizer subgroup generated as: \[ \langle ZIIIZII, IZIIZII, IIZIIZI, IIIZIIZ, IIXXIYY, XXXZXXI \rangle \]
Label the generators as $g_1, \cdots, g_6$ according the ordering presented above.
\begin{enumerate}
	\item To show that this code can correct any single qubit error, it suffices to show that all Pauli operators with weight at most two anti-commute with at least one of the generators $g_i$ or they lie inside the above generated stabilizer subgroup. 
		\begin{description}
			\item[Weight 1 Paulis] Each weight one Pauli operator $X_i, Y_i, Z_i$ much anti-commute with at least one generator $g_i$. This can be immediately seen by observing that for each weight one operator, there exists at least one generator with a differing non-identity Pauli at each index.
			\item[Two $Z$ Operators] With a Pauli operator $P$ comprising of precisely two $Z$ operators, it trivially commutes with $g_1, \cdots g_4$. However, by iterating through the possibilties, we can conclude that the operator cannot commute with both $g_5,g_6$ unless \[ P=Z_1Z_2, Z_1Z_5, Z_2Z_5, Z_3Z_6, Z_4Z_7 \]
				However, all these cases are generated by $g_1, \cdots g_4$.
			\item[Two $X$ Operators] When $P$ is a weight two Pauli operator comprised of exactly two $X$ operators in differing indices, we can directly calculate that the $P$ which commute with both $g_5, g_6$ will be 
				\begin{align*}
					P = X_5X_6, X_1X_2, X_1X_3, X_2X_3, X_2X_5, X_1X_5, X_3X_5
				\end{align*}
				Each of these operators anti-commutes with at least one generator in $g_1, g_2, g_3$
				Alternatively, we could argue that for every distinct pair of indices $1 \leq i < j \leq 7$, there exists at least one generator amongst the $g_i$ such that, when restricting the total tensor to indices $i,j$, the result is equal to one of the following:
		\begin{enumerate}
			\item \(Z_i \otimes I_j, I_i \otimes Z_j\) 
			\item  \(Y_i \otimes I_j, I_i \otimes Y_j\) 
			\item \(Z_i \otimes X_j, X_i \otimes Z_j\) 
			\item  \(Y_i \otimes X_j, X_i \otimes Y_j\) 
		\end{enumerate} 
			\item[Two $Y$ Operators] This case can be considered by noticing that for every distinct pair of indices $1 \leq i < j \leq 7$, there exists at least one generator such that, when restricting the total tensor to indices $i,j$, the result is equal to one of the following:
				\begin{enumerate}
					\item $X_i \otimes I_j, I_i \otimes X_j$
					\item  $Z_i \otimes I_j, I_i \otimes Z_j$
					\item $X_i \otimes Y_j, Y_i \otimes X_j$
					\item  $Z_i \otimes Y_j, Y_i \otimes Z_j$
				\end{enumerate}
				showing that $P$ must anti-commute with at least one generator $g_i$.
			\item[$X,Z$ Operator] In the case where $P$ contains one $X$ and one $Z$ operator, note that for generators $g_1, \cdots g_4$, and an index $1 \leq i \leq 7$, there exists at least one $g_h$ where $Z_i$ is contained in $g_h$. Hence, for every pair of $X,Z$ operators. The $X$ anti-commutes with one of the $Z_i$ and the $Z$ trivially commutes with itself and the identity. This shows that $P$ must commute with one of the $g_1, \cdots, g_4$.
				%We can divide the analysis into cases:
				%\begin{itemize}
					% \item The only generator which contains both a $X,Z$ is $g_6$ at indices $1,4$. However, any $P$ with its $X,Z$ placed in either index will anti-commute with one of the $g_1, \cdots g_4$.
					% \item If $x_i = 1,2,4, 5,6,7$, then the $P$ will anti-commute with one of the $g_1,g_2,g_3,g_4$.
				 %	\item If $x_i = 3$, then $P$ will commute with $g_6$ as long as $z_i =  4, 7$. However, this would inevitably mean that $P$ anti-commutes with $g_5$.
					% \item $x_i = 4$, then $P$ will commute with $g_6$ as long as $z_i \neq 7$. However, once again, this subcase will force $P$ to anti-commute with $g_5$. 

				%\end{itemize}

			\item[$Y,Z$ operators] This case follows identitically as the case above.
			\item[$X,Y$ operators] A Pauli operator $P$ consisting of an $X,Y$ operator pair. We consider the case where $P$ is of the form $X_i \otimes Y_j$ for $i < j$. We employ a similar strategy as above to note that the at least one of the generators, when restricted to its indices $i,j$ from the total tensor product, will equal to one of the following:
				\begin{enumerate}
					\item $Z_i \otimes I_j, I_i \otimes Z_j$
					\item $Z_i \otimes Y_j, X_i \otimes Z_j$
					\item $Y_i \otimes I_j, Y_i \otimes Y_j$
					\item $I_i \otimes X_j, X_i \otimes X_j$
				\end{enumerate}
		\end{description}
	

	\item The logical operators for this code can be defined as:
		\begin{align*}
			\overline{X} & = X_1X_2X_5 = XXIIXII \\
			\overline{Z} & = Z_5Z_6Z_7 = IIIIZZZ 
		\end{align*}

		Indeed, both operators do not lie in the stabilizer subgroup, commute with all generators $g_i$, and the operators anti-commute as required:
		\begin{equation*}
			\overline{X}\overline{Z} = - \overline{Z} \overline{X}
		\end{equation*}

	\item The calculations above show that the code has distance $d = 3$. When conjugating according to single qubit Hadamard gates $H_i$, $S_i$, the analysis of parts one and two taken in respect to the conjugated generators, as well the corresponding conjugated logical operators, will be similar to the ones presented above. For example, in the case of the single-qubit Hadamard operators $H_j$ for some $1 \leq j \leq 7$, we know that the Hadamard operator adheres to the conjugation actions:
		\begin{align*}
			& H_jX_jH_j = Z_j \\
			& H_jZ_jH_j = X_j \\
			& H_jY_jH_j = -Y_j
		\end{align*}
		Conjugating the generators $g_1,\cdots g_6$ according to $H_j$ will yield another set of generators $H_jg_1H_j, \cdots H_jg_6H_j$ which commute with each other. The logical operators of the conjugated stabilizer subgroup will become $H_j\overline{X}H_j, H_j\overline{Z}H_j$. The cases considered in part one will hold as long as we perform the appropriate subsitutions in the argument: if some Pauli $P$ anti-commutes with generator $g_i$, then:
		\begin{equation*}
			H_jPH_j\cdot H_jg_iH_j = H_j P g_i H_j = - H_j g_i P H_j = - H_jg_i H_j \cdot H_j P H_j
		\end{equation*}
		(Recall that $H_j^\dagger = H_j$) \newline
		A simlar argument holds for the phase gate $S = \left[ \begin{matrix} 1 & 0 \\ 0 & i \end{matrix} \right]$. This procedure would ultimately show that the conjugated code also has distance $d = 3$, hence it can correct any single qubit error.
		\end{enumerate}
\end{document}


