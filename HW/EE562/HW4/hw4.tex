\documentclass[12pt]{article}%
\usepackage{amsfonts}
\usepackage{amsmath}
\usepackage{amssymb}

\usepackage{geometry}
\usepackage{physics}

\usepackage{times}
\usepackage{microtype}


\usepackage{mathtools}
\usepackage{hyperref} % backref=page
\usepackage[shortlabels]{enumitem}

% Prints a trailing space in a smart way.
\usepackage{xspace}
\usepackage[font=footnotesize]{caption}
\usepackage{titlesec}

\DeclarePairedDelimiter\ceil{\lceil}{\rceil}
\DeclarePairedDelimiter\floor{\lfloor}{\rfloor}
\newenvironment{proof}[1][Proof]{\textbf{#1.} }{\ \rule{0.5em}{0.5em}}


% General Mathematical Notation
\newcommand{\reals}{{\mathbb{R}}}
\newcommand{\complex}{{\mathbb{C}}}                    %reals                  %reals
\newcommand{\nnreal}{{\nnnum R}}
                   %nonnegative

\newcommand{\nat}{\mathbb{N}}                      %natural numbers
\newcommand{\integers}{{\num Z}}                      %integers
\newcommand{\rat}{{\num Q}}                      %rationals
\newcommand{\nnrat}{{\nnnum Q}}

\newcommand{\boldu}{\textbf{u}}
\newcommand{\boldv}{\textbf{v}}
%\newcommand{\ip}[1][2]{\langle {#1}, {#2},\rangle}

% Differentiation
\newcommand{\totder}[1][]{\frac{d^{#1}}{dx^{#1}}}

% Environments
\newtheorem{theorem}{Theorem}[section]
\newtheorem{lemma}{Lemma}[section]

% Series
\newcommand{\powaseries}[1][0]{\sum_{n = {#1}}^{\infty}}

% Vector Spaces

% Hilbert Space
\newcommand{\hil}{\mathcal{H}}
\newcommand{\bhilop}{\mathcal{B}(\mathcal{H})}


\geometry{letterpaper,left=1in,top=1in,right=1in,bottom=1in, footskip=0.5in, nohead}



% Misc
\newcommand{\parafrac}[2]{\para{\frac{#1}{#2}}}

%HW 10
\newcommand{\wc}{w(\mathcal{C})}
\newcommand{\C}{\mathcal{C}}
\newcommand{\Lcal}{\mathcal{L}}
\newcommand{\Cep}{\mathcal{C}_\epsilon}
\newcommand{\CR}{\mathcal{C}_R}
\newcommand{\sgn}[1]{\text{sgn}({#1}) }




\title{EE562 HW 4}
\author{Edward Kim}
\date{\today}

\begin{document}
\maketitle
\section{Problem 1 (Hajek 3.9)}
\begin{enumerate}
  \item By definition, we see that $\Exp{X\mid U = u} = \frac{1}{1 + u}$ for $0 \leq u \leq 1$. Consequently, $\Exp{X \mid U} = \frac{1}{1 + U} = X$. The mean-square error (MSE) is readily calculated to be:
  \[ \Exp{e^2} = \Exp{X^2} - \Exp{(\Exp{X \mid U})^2} = \Exp{X^2} - \Exp{X^2}  = 0  \]

  This aligns with the observation that $h(U) = \frac{1}{1 + U}$ is a function dependent on the random variable $U$. Thus, $h(U) \in \mathcal{V}$ where $\mathcal{V}$ is the closed linear subspace of $L^2(\Omega,\mathcal{F},\mathbb{P})$ of unconstrained estimators dependent on $U$ and, in particular, $X \in \mathcal{V}$.

  \item We find $\LinEsti{X}{U}$ by invoking the explicit form of the linear estimator of a random variable:

  \begin{equation}
 \LinEsti{X}{U} = \Exp{X} + \Cov{X}{U}\cdot\Var{U}^{-1}\left(U - \Exp{U}\right) \label{eq:one}
  \end{equation}

  Finding the required values is straightforward:
  \begin{align*}
      \Exp{U^2} & = \frac{1}{3}, \quad \Var{U} = \frac{1}{12} \\
      \\
      \Cov{X}{U} & = \Exp{XU} - \Exp{X}\Exp{U} \\
                 & = (1 - \ln{2}) - \frac{\ln{2}}{2} \\
                 &= \frac{2 - 3\ln{2}}{2}
  \end{align*}

  Hence, by \ref{eq:one},
  \begin{align*}
      \LinEsti{X}{U} & =  \ln{2} + 12 \cdot \frac{2 - 3\ln{2}}{2}\cdot\left(U - \frac{1}{2}\right)\\
      & =  \ln{2} + (12 - 18\ln{2})\cdot\left(U - \frac{1}{2}\right)
  \end{align*}
  %
  The MSE is once again computed through orthogonality:

  \begin{align*}
    \Exp{e^2} &= \Exp{X^2} - \Exp{(\LinEsti{X}{U})^2} \\
    & = \frac{1}{2} - \Exp{\left( \ln{2} + (12 - 18\ln{2})\cdot\left(U - \frac{1}{2}\right)\right)^2} \\
    & = \frac{1}{2} - \left(\ln^2{2} + (12 - 18\ln{2})^2\cdot \Exp{\left(U - \frac{1}{2}\right)^2}\right)
  \end{align*}

  The expectation on the right is ascertained through previous intermediate calculations used above to be $\Exp{\left(U - \frac{1}{2}\right)^2} = 1/12$. The total expression for the MSE can be simplified to give the numerical approximation:

  \[ \Exp{e^2} \approx 0.001 \]
\end{enumerate}

\section{Problem 2 (Hajek 3.19)}
   Our quadratic estimator will be the sum of the terms of $\xhat = a + bY + cY^2$ for $a,b,c \in \reals$ and we will be projecting on the subspace of quadratic estimators with input $Y$. Denote this subspace as $\mathcal{V}$ for this problem.
\begin{enumerate}
  \item
  Like the linear estimator case, we will exploit the constraints imposed by the orthogonality conditions. Let $e = X - \xhat$ be the error variable. The orthogonality conditions can be succinctly expressed as follows:
  \begin{align}
      e & \perp 1 \iff \Exp{e} = 0 \label{eq:cond1}\\
      e & \perp Y \iff \Cov{e}{Y} = 0 \label{eq:cond2}\\
      e & \perp Y^2 \iff \Cov{e}{Y^2} = 0 \label{eq:cond3}
  \end{align}

  With these constraints, we can begin to solve for the coefficient factors $a,b,c$.

  \begin{align*}
    \Exp{e} & = \Exp{X - (a + bY + cY^2)} \\
            & = \Exp{X} - a - c
  \end{align*}
  By the estimates given in the book and condition \ref{eq:cond1}, we arrive at the linear relation

\begin{equation}
  a + c = 4/5 \label{eq:aconstantconst}
\end{equation}

  Continuing forth, we now exploit conditions \ref{eq:cond2},\ref{eq:cond3}:
  %
  \begin{align*}
    \Cov(e,Y) & = \Cov{X}{Y} - \Cov{bY}{Y} - \Cov{cY^2}{Y} \\
      & = \Cov{X}{Y} - b\cdot\Var{Y} - c\cdot\Cov{Y^2}{Y}
  \end{align*}
  However, arguing on the symmetry of the standard normal distribution in respect to the origin, $\Exp{Y^3} = 0$ and $\Cov{X}{Y} = \Exp{\abs{Y}\cdot Y} = 0$. This shows that $b = 0$.

  Finally, the last condition \ref{eq:cond3} imposes that:

  \begin{align*}
    \Cov{e}{Y^2} = \Cov{X}{Y^2} - b\cdot \Cov{Y}{Y^2} - c\cdot \Cov{Y^2}{Y^2} = 0
  \end{align*}

  The relevant covariances are directly determined to be:
  \begin{align*}
      \Cov{X}{Y^2} & = \Exp{\abs{Y}\cdot Y^2} - \Exp{\abs{Y}}\Exp{Y^2} \\
      & = 1.6 - 0.8\cdot 1 \\
      & = \boxed{4/5}
      \\
      \\
      \Cov{Y^2}{Y^2} & = \Exp{Y^4} - (\Exp{Y^2})^2 \\
                     & = \boxed{2}
  \end{align*}

  Substituting these values into the constraint above and simplifying after taking expectations yields that:

  \[ \frac{4}{5} - 2c = 0 \implies \boxed{x = \frac{2}{5}} \]
  The final constant $a$ is determined by linear relation \ref{eq:aconstantconst}, showing that $a = \frac{2}{5}$

  \item In light of the work shown above, our estimator will be of the form:
  \[ \xhat = \frac{2}{5} + \frac{2}{5}Y^2 \]

  \item
  The MSE will become:
  \begin{align*}
    \Exp{e^2} & = \Exp{X^2} - \Exp{\left(\frac{2}{5} + \frac{2}{5}Y^2\right)^2} \\
    & = 1 - \left(\frac{2}{5}\right)^2 - 2\left(\frac{2}{5}\right)^2 \Exp{Y^2} - \left(\frac{2}{5}\right)^2 \Exp{Y^4} \\
    & = 1 - \frac{4}{25} - \frac{8}{25} - \frac{12}{25}\\
    & = \boxed{\frac{1}{25}}
  \end{align*}
\end{enumerate}

\section{Problem 3}
We first turn our attention to the unconstrained estimator based on $Y$. First, note that the expectations conditioned on $Y=1$ can be readily found through direct calculation:

\begin{align*}
  \CondExp{X}{Y = 1} & = 1\cdot \frac{\Prob{X = 1 \cap Y = 1)}}{\Prob{Y = 1}} + 3 \cdot \frac{\Prob{X = 3 \cap Y = 1)}}{\Prob{Y = 1}} + 5 \cdot \frac{\Prob{X = 5 \cap Y = 1)}}{\Prob{Y = 1}} \\
  & = (1 + 3 + 5) \cdot \frac{1}{3} \\
  & = \boxed{3}
\end{align*}

Similarly, $\boxed{\CondExp{X}{Y = 2} = 4}$. These two cases demonstate that the conditional expectation can be expressed as

\begin{equation}
  \CondExp{X}{Y} = Y + 2 \label{eq:prob3_unconstresti}
\end{equation}
Additionally, the MSE error will be:

\begin{align*}
  \Exp{X^2} - \Exp{\left(Y + 2\right)^2} & = \frac{7 \cdot 13}{6} - (4 - 2\Exp{Y} + \Exp{Y^2}) \\
  & =\frac{7 \cdot 13}{6} - 4 - 3 - \frac{5}{2} \\
  & = \boxed{\frac{34}{6}}
\end{align*}

Consider now the case where we optimize over all linear estimators based on $Y$:

\begin{align}
  \LinEsti{X}{Y} = \Exp{X} + \Cov{X}{Y}\Var{Y}^{-1}\left(Y - \Exp{Y}\right) \label{eq:prob3_linesti}
\end{align}
The covariances are found to be:
\begin{align*}
\Var{Y} & = \frac{5}{2} - \left(\frac{3}{2}\right)^2 = \boxed{\frac{1}{4}} \\
\\
\Cov{X}{Y} & = \Exp{XY} - \Exp{X}\Exp{Y} \\
           & = \left((1+3+5)\cdot \frac{1}{6} + (2 + 4 + 6)\cdot 2 \cdot \frac{1}{6}\right) \\
           & = \boxed{\frac{1}{4}}
\end{align*}

Once again, substitute the covariances into \ref{eq:prob3_linesti} to find the linear estimator of $X$ based on $Y$:

\begin{align}
  \LinEsti{X}{Y} & = \frac{7}{2} + \left(Y - \frac{3}{2}\right) \\
          & = Y + 2
\end{align}

which is exactly the unconstrained estimator found in \ref{eq:prob3_unconstresti}. Thus, we deduce that the most optimal unconstrained estimator of $X$ given $Y$ is actually linear.

\section{Problem 4 (Hajek 3.7(a))}
By virtue of orthogonality, \[\Exp{X^2} = \Exp{e^2} + \Exp{W^2}\]
where we set $W = \CondExp{X}{Y}$. From the information given in the mean vector and covariance matrix, we know that $\Var{X} = 8$ and $\Exp{X} = 2$. Thus, $\Exp{X^2} = 12$. However without additional information on the joint PDF of $X,Y$ or additional structure (e.g $X,Y$ are jointly Gaussian), the value of $\Exp{W^2}$ would be indeterminable. Without an exact value, we can bound $\Exp{e^2}\leq 12$.

\section{Problem 5 (Hajek 3.5)}
\begin{enumerate}
  \item We first determine the conditional density function $\condPDF{X}{Y}$:

  \def \xyvec{\begin{bmatrix} x \\ y \end{bmatrix}}
  \def \GaussExpon{e^{- \frac{1}{2} Q(x,y)}}
  \def \JointDen{\frac{1}{2\pi(\sqrt{1-\rho^2})}\cdot \GaussExpon}
  \def \MarginalDen{{\frac{1}{\sqrt{2\pi}} e^{-\frac{y^2}{2}}}}

  \begin{align*}
      \condPDF{X}{Y} & = \frac{\jointPDF{X}{Y}}{f_Y(y)} \\
                     & = \frac{\JointDen}{\MarginalDen}
  \end{align*}

  where $Q(x,y) = \xyvec^T K^{-1} \xyvec$ and $K^{-1} = \frac{1}{1 - \rho^2} \begin{bmatrix}
    1 & -\rho \\ -\rho & 1
  \end{bmatrix}$. Now expanding out $Q(x,y)$ in the exponent of the joint Gaussian allows up to simplify and massage our expression into another Gaussian:
  %
  \begin{align*}
  \condPDF{X}{Y} & = \frac{1}{\sqrt{2\pi(1-\rho^2)}} e^{-\frac{1}{2(1-\rho^2)} (x^2 - 2\rho xy + \rho^2y^2)} \\
  & = \frac{1}{\sqrt{2\pi(1-\rho^2)}} e^{-\frac{1}{2}\cdot \frac{(x - \rho y)^2}{(1 - \rho^2)}}
  \end{align*}

  It is now immediate that $\Prob{X \leq 1 \mid Y} = \Phi(\frac{1 - \rho Y}{\sqrt{1 - \rho^2}})$

  \item
  Recall that both marginal distributions of the standard joint normal distribution between a pair of random variables will be precisely $\standardnormdist$. In light of this fact:
  \begin{align*}
    \CondExp{(X-Y)^2}{Y=1} & = \Exp{(X-y)^2} \\
                           & = \Exp{X^2} - 2y\Exp{X} + y^2 \\
                           & =  1 + y^2
  \end{align*}
  where we have used $\Exp{X} = 0, \; \Var{X} = 1$.
\end{enumerate}

\section{Problem 6}
\begin{enumerate}
  \item The intuition here is that if $X \sim \standardnormdist$, then $-X$ is identically distributed as $X$ by symmetry. Thus, $Y$ should give us another standard Gaussian as a uniformly weighted sum between two identically distributed standard Gaussian variables. To this end, we first compute the values of the CDF of $Y$ as below:

  \begin{align*}
    \Prob{Y \leq y} & = \frac{1}{2}\cdot \Phi(y) + \frac{1}{2}\cdot (1 - \Phi(-y)) \\
    & = \frac{1}{2} +\frac{\Phi(y) - \Phi(-y)}{2} \\
    & = \frac{1}{2} +\frac{1}{2} \left(\frac{1}{\sqrt{2\pi}} \int_{-y}^y e^{-v^2/2}\; dv\right) \\
    & = \frac{1}{\sqrt{2\pi}} \int^{0}_{-\infty} e^{-v^2/2} \; dv + \frac{1}{\sqrt{2\pi}} \int_{0}^y e^{-v^2/2} \; dv \\
    & = \frac{1}{\sqrt{2\pi}} \int_{-\infty}^{y} e^{-v^2/2} \; dv
  \end{align*}
  which shows that indeed $Y = SX$ is distributed according to $\standardnormdist$ as desired.

  \item On the other hand, consider the sum $X + Y = X(1+S)$. Through definition, $X+Y$ takes values $2X$ and $0$ with equal probability. This is clearly not a Gaussian variable, as there is a sharp discontiunity in the PDF $f_{X+Y}$ arising from the probability $1/2$ of the sum vanishing completely. This shows that finite linear combinations of Gaussian random variables need not be Gaussian if the family is \emph{not} independent.
\end{enumerate}

\section{Problem 7 (Hajek 3.25)}
As according to the problem, set $W = X + 2Y + 3$
\begin{enumerate}
  \item
  \begin{align}
    \Exp{W} & = \Exp{X} + 2\Exp{Y} + 3 = 2 + 8 + 3 = \boxed{13} \\
    \Var{W} & = \Var{X} + 4\cdot\Var{Y} + 4\cdot\Cov{X}{Y} \\
    & = 9 + 4\cdot 25 + 4 \cdot 0.2 \\
    & = 109.8
  \end{align}

  \item
  As $X,Y$ are jointly Gaussian, $W$ must be Gaussian and thus must be distributed according to $\normdist{13}{109.8}$. Using any table of numerical values:
  $\Prob{W \geq 20} \approx 0.475$

  \item
  From Proposition 3.9 from the textbook, we know that since $Y,W$ are jointly Gaussian (as finite linear combinations of joint Gaussian variables $X,Y$), the conditional distribution of $Y$ based on $W = w$ is distributed according to $\normdist{\LinEsti{Y}{W = w}}{\Cov{e}{e}}$. In particular, the taking the expectation over this conditional distribution shows that $\CondExp{Y}{W} = \LinEsti{Y}{W}$. Thus, for jointly Gaussian variables $Y,W$, our unconstrained estimator of $Y$ based on $W$ is exactly given by the linear estimator. We proceed to compute this linear estimator:

  \begin{align*}
    \LinEsti{Y}{W} & = \Exp{Y} + \Cov{Y}{W}\Var{W}^{-1}\left(W - \Exp{W}\right) \\
    & = 4  + \frac{251}{549}(W - 13)
  \end{align*}
  as $\Var{W} = 109.8$ and $\Cov{Y}{W} = \Cov{Y}{X} + 2\cdot \Cov{Y}{Y} = 0.2 + 2 \cdot 25 = 50.2$.

  As for the MSE, this can be readily computed as:
  \begin{align*}
    \Exp{e^2} & = \Exp{Y^2} - \Exp{(4 +  \frac{251}{549}(W - 13))^2} \\
    & = 41 - \left[16 +  \left(\frac{251}{549}\right)^2\Var{W}\right] \\
    & \approx 2.049
  \end{align*}
\end{enumerate}


\end{document}
