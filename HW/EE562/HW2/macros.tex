\DeclarePairedDelimiter\ceil{\lceil}{\rceil}
\DeclarePairedDelimiter\floor{\lfloor}{\rfloor}
\newenvironment{proof}[1][Proof]{\textbf{#1.} }{\ \rule{0.5em}{0.5em}}


% General Mathematical Notation
\newcommand{\reals}{{\mathbb{R}}}
\newcommand{\complex}{{\mathbb{C}}}                    %reals                  %reals
\newcommand{\nnreal}{{\nnnum R}}
                   %nonnegative

\newcommand{\nat}{\mathbb{N}}                      %natural numbers
\newcommand{\integers}{{\num Z}}                      %integers
\newcommand{\rat}{{\num Q}}                      %rationals
\newcommand{\nnrat}{{\nnnum Q}}

\newcommand{\boldu}{\textbf{u}}
\newcommand{\boldv}{\textbf{v}}
%\newcommand{\ip}[1][2]{\langle {#1}, {#2},\rangle}

% Differentiation
\newcommand{\totderop}[1][]{\frac{d^{#1}}{dx^{#1}}}
\newcommand{\partder}[2]{\frac{\partial{#1}}{\partial{#2}}}
\newcommand{\totder}[2]{\frac{d{#1}}{d{#2}}}

% Integration
\newcommand{\doubinfint}{\int_{-\infty}^{\infty}}

% Environments
\newtheorem{theorem}{Theorem}[section]
\newtheorem{lemma}{Lemma}[section]

% Series
\newcommand{\powaseries}[1][0]{\sum_{n = {#1}}^{\infty}}

% Vector Spaces

% Hilbert Space
\newcommand{\hil}{\mathcal{H}}
\newcommand{\bhilop}{\mathcal{B}(\mathcal{H})}

% Distributions
\newcommand{\testfunc}{C_F^{\infty}(\reals)}
\newcommand{\prinval}[1]{P\left({#1}\right)}

% Typesetting
\newcommand{\para}[1]{\left( {#1} \right)}
% \newcommand{\brack}[1]{\left[ {#1} \right]}

\newcommand{\Uone}{\mathcal{U}_1}
\newcommand{\Utwo}{\mathcal{U}_2}
\newcommand{\Uint}{\Uone \cap \Utwo}
\newcommand{\V}{\mathcal{V}}
\newcommand{\gpart}[1][t]{\frac{\partial g(x,t)}{\partial {#1}}}
\setlength\parindent{0pt}
\newcommand{\Legen}[1][n]{(x^2 - 1)^{#1}}

\newcommand{\normphi}[1][v]{\abs{\Phi\left({#1}
\right)}}

\newcommand{\disip}[2]{\left\langle #1, #2 \right\rangle}
\newcommand{\infsum}[1][0]{\sum^{\infty}_{n = {#1}}}
\newcommand{\posinfint}[1][0]{\int_{#1}^\infty}
\newcommand{\neginfint}[1][0]{\int_{-\infty}^{#1}}

% Fourier Shortcuts
\newcommand{\piinter}{[-\pi,\pi]}
\newcommand{\pifrac}{\frac{1}{\pi}}
\newcommand{\piint}[2][x]{\int_{-\pi}^\pi {#2} \; d{#1}}
\newcommand{\twopiint}[2][x]{\int_{0}^{2\pi} {#2} \; d{#1}}
\newcommand{\foutranscoeff}[1][k]{\tilde{f}({#1})}
\newcommand{\foutrans}[1]{\mathcal{F}\left[{#1}\right](k)}

% Complex Analysis
\newcommand{\contint}[1][\mathcal{C}]{\oint_{#1}} %Contour integral over integral curve C


% Probability
\newcommand{\Prob}[1]{\mathbb{P}[\,#1\,]} % mathbb probability symbol
\newcommand{\Exp}[1]{\mathbb{E}\left[\,#1\,\right]} % mathbb expectation operator
\newcommand{\asconv}{\xrightarrow[]{a.s}} % almost surely convergence rightarrow
\newcommand{\probconv}{\xrightarrow[]{p}} % convergence in probability rightarrow
\newcommand{\msconv}{\xrightarrow[]{m.s}} % mean square convergence rightarrow
\newcommand{\distconv}{\xrightarrow[]{d}} % mean square convergence rightarrow
