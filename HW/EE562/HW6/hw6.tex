\documentclass[12pt]{article}%
\usepackage{amsfonts}
\usepackage{amsmath}
\usepackage{amssymb}

\usepackage{geometry}
\usepackage{physics}

\usepackage{times}
\usepackage{microtype}


\usepackage{mathtools}
\usepackage{hyperref} % backref=page
\usepackage[shortlabels]{enumitem}

% Prints a trailing space in a smart way.
\usepackage{xspace}
\usepackage[font=footnotesize]{caption}
\usepackage{titlesec}

\DeclarePairedDelimiter\ceil{\lceil}{\rceil}
\DeclarePairedDelimiter\floor{\lfloor}{\rfloor}
\newenvironment{proof}[1][Proof]{\textbf{#1.} }{\ \rule{0.5em}{0.5em}}


% General Mathematical Notation
\newcommand{\reals}{{\mathbb{R}}}
\newcommand{\complex}{{\mathbb{C}}}                    %reals                  %reals
\newcommand{\nnreal}{{\nnnum R}}
                   %nonnegative

\newcommand{\nat}{\mathbb{N}}                      %natural numbers
\newcommand{\integers}{{\num Z}}                      %integers
\newcommand{\rat}{{\num Q}}                      %rationals
\newcommand{\nnrat}{{\nnnum Q}}

\newcommand{\boldu}{\textbf{u}}
\newcommand{\boldv}{\textbf{v}}
%\newcommand{\ip}[1][2]{\langle {#1}, {#2},\rangle}

% Differentiation
\newcommand{\totder}[1][]{\frac{d^{#1}}{dx^{#1}}}

% Environments
\newtheorem{theorem}{Theorem}[section]
\newtheorem{lemma}{Lemma}[section]

% Series
\newcommand{\powaseries}[1][0]{\sum_{n = {#1}}^{\infty}}

% Vector Spaces

% Hilbert Space
\newcommand{\hil}{\mathcal{H}}
\newcommand{\bhilop}{\mathcal{B}(\mathcal{H})}


\geometry{letterpaper,left=1in,top=1in,right=1in,bottom=1in, footskip=0.5in, nohead}



% Misc
\newcommand{\parafrac}[2]{\para{\frac{#1}{#2}}}

%HW 10
\newcommand{\wc}{w(\mathcal{C})}
\newcommand{\C}{\mathcal{C}}
\newcommand{\Lcal}{\mathcal{L}}
\newcommand{\Cep}{\mathcal{C}_\epsilon}
\newcommand{\CR}{\mathcal{C}_R}
\newcommand{\sgn}[1]{\text{sgn}({#1}) }




\title{EE562 HW 6}
\author{Edward Kim}
\date{\today}

\begin{document}
\maketitle

\section{Problem 1}

We verify the required criteria. That the process $Z$ is of second-order is evident by the Cauchy-Schwarz Inequality:
%
\[\abs{\Exp{X(t)\cdot Y(t)}}^2 \leq \Exp{\abs
{X(t)}^2} \Exp{\abs{Y(t)}^2} < \infty \]
for all $t$ as both $X(t),Y(t)$ are WSS. By independence, the remaining two criteria are fulfilled:

\begin{align*}
  \Exp{Z(t)} & = \Exp{X(t)}\cdot\Exp{Y(t)} \\
  \Exp{Z(t)\cdot Z(s)} & = \Exp{X(t)\cdot Y(t)}\times \Exp{X(s)\cdot Y(s)}
\end{align*}

Once again, the products of expectations of the random varables $X(t),Y(t)$ for any fixed $t$ will remain constant as both $X,Y$ are WSS processes. This directly shows that $Z = X \cdot Y$ is indeed a WSS process as well.

\section{Problem 2}
\begin{enumerate}
  \item It is evident that, as a Gaussian process, the $X_t$ will be distributed according to some Gaussian distribution. From the correlation matrix, it is clear that $X_t \sim \normdist{0}{5}$.

  \item
  The joint distribution between $X(t),X(t+1)$ can be deduced from the joint density:
%
  \begin{align*}
    f_{X(t),X(t+1)}({\bold x}) = \frac{1}{2\pi \sqrt{\det K}} \cdot e^{-\frac{1}{2}({\bold x} K^{-1} {\bold x}^T)}
  \end{align*}
  where ${\bold x} = (x_1,x_2)$ is a row vector in this context. The relevant values can be computed directly:

  \begin{align*}
    K^{-1} = \frac{1}{25(1 - e^{-4})}\begin{bmatrix}
        5 & 5e^{-2} \\  5e^{-2} & 5
  \end{bmatrix}
  \end{align*}
\end{enumerate}

\section{Problem 3}
\begin{enumerate}
  \item By Proposition 4.5(c), the joint density of the arrival times conditioned on counting process $N(t)$ for some fixed $t$ will be:

\[ f_{T_1,\cdots,T_n\mid N(t)}(t_1,\cdots,t_n\mid N(t) = n) = \begin{cases}
    \frac{n!}{t^{n}} \quad 0 < t_1 < t_2 < \cdots < t_n \leq t \\
    0 \quad \text{otherwise}
\end{cases} \]

  In light of this formula, it is evident that:

  \[ f_{T_1,T_2\mid N(2)}(t_1,t_2 \mid N(2) = 2) = \begin{cases}
      \frac{1}{2} \quad 0 < t_1 < t_2  \leq t \\
      0 \quad \text{otherwise}
  \end{cases} \]

  The derivation concludes by properly integrating over $t_2$ to find the marginal density, and integrating once more:

  \begin{align*}
     f_{T_1\mid N(2)}(t_1\mid N(2)=2) & = \int_0^{t_1} f_{T_1,T_2\mid N(2)}(t'_1,t'_2 \mid N(2) = 2) \; dt'_2 \\
    & = \frac{t_1}{2}
    \end{align*}

  Subsequently,
  \begin{align*}
    \Prob{N(1) \geq 1 \mid N(2) = 2} &  = 1 - \int_{0}^1 \frac{t_1}{2}\; dt_1 \\
    & = 1 - \frac{1}{4} \\
    & = \boxed{\frac{3}{4}}
  \end{align*}

  \item
  Invoking Bayes Rule and the value computed in the previous part of this problem yields the desired value:

  \begin{align}
    \Prob{N(2) = 2  \mid N(1) \geq 1} & = \frac{\Prob{N(1) \geq 1 \mid N(2) = 2}\cdot\Prob{N(2)=2}}{\Prob{N(1) \geq 1}} \\
    & = \frac{3}{4}\cdot \frac{\frac{(2\lambda)^2}{2!}\cdot e^{-2\lambda}}{e^{-\lambda}} \\
    & = \frac{3}{2}\lambda^2e^{-\lambda}
  \end{align}

\end{enumerate}

\section{Problem 4}
\begin{enumerate}
  \item Yes. The process $X$ is stationary. As the $U_n$ are independent and identically distributed, a case analysis reveals that the distributions of the nth order cumulative functions $F_{X,n}(x_1,t_1;\cdots;x_n,t_n)$ will only depend on the relative distances between the time points $t_1,\cdots ,t_n$.

  To see this, we restrict ourselves to the $n=2$ case as the general case will immediately extend from it. If $\abs{t_2 - t_1} \geq 2$, then it is evident that $F_{X,2}(x_1,t_1 ; x_2, t_2) = x_1^2x_2^2$. Otherwise, if $\abs{t_2 - t_1} = 1$, then $F_{X,2}(x_1,t_1 ; x_2, t_2) = x_1x_2 \min\{x_1,x_2\}$. Finally, if $t_1 = t_2$, then $F_{X,2}(x_1,t_1 ; x_2, t_2) = \min\{x_1^2,x_2^2\}$. In particular, none of the probabilities above depend on the time points $t_1,t_2$. The argument can be generalized to the nth order by repeating it for valid pairs $(t_i, t_{i+1})$ constructed from $0 < t_1 < t_2 < \cdots < t_n$.

  % Furthermore, the joint distribution will split into its constituent marginal distributions:
  %\[F_{X,n}(x_1,t_1;\cdots;x_n,t_n) = \Prob{X_{t_1} \leq x_1, \cdots, X_{t_n} \leq x_n} \]

  \item

  \item

\end{enumerate}

\section{Problem 5}
\begin{enumerate}
  \item Begin by rearranging the terms of the inequality defining the event and re-expressing the Brownian motion random variables in terms of their independent increments:
  \begin{align*}
\Prob{W_3 \geq \frac{W_2 + W_4}{2} + 1} & = \Prob{2W_3 - (W_2 + W_4) \geq 2} \\
& = \Prob{(W_3 - W_2) - (W_4 - W_3) \geq 2}
  \end{align*}

  As the increments are independent, the random variable on the left-hand side will be a linear combination of two standard normal distribution. Hence, it will distributed according to $\normdist{0}{2}$. The probability above can now be stated in terms of the $Q$ function:
  \begin{align*}
    \Prob{W_3 \geq \frac{W_2 + W_4}{2} + 1} & = Q(2/\sqrt{2}) \\
    & = Q(\sqrt{2})
  \end{align*}

  \item Recall that the distribution of $X^2$, where $X \sim \normdist{0}{\sigma^2}$, can be expressed as:

  \[F_{X^2}(x) = \Prob{\abs{X} \leq \sqrt{x}} = \Phi\left(\frac{\sqrt{x}}{\sigma}\right) - \Phi\left(-\frac{\sqrt{x}}{\sigma}\right) \]

  Define $X_t = \frac{W_t^2}{t}$. Since $W_t \sim \normdist{0}{t}$, we can harness the observation above to compute the limiting distribution:

  \begin{align*}
    F_{X_t}(x) & = \Prob{X_t \leq x} \\
      & = \Prob{W_t \leq \sqrt{xt}} \\
      & = \Phi\left(\frac{\sqrt{xt}}{\sqrt{t}}\right) - \Phi\left(-\frac{\sqrt{x}}{\sqrt{t}}\right)  \\
      & = \Phi(\sqrt{x}) - \Phi(-\sqrt{x}) \\
  \end{align*}

  All together, the derivations imply that $F_{X_t}(x) \distconv \widetilde{X}^2$ where $\widetilde{X} \sim \standardnormdist$.

\end{enumerate}


\end{document}
