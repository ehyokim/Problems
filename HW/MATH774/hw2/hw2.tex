\documentclass[12pt]{article}%
\usepackage{amsfonts}
\usepackage{amsmath}
\usepackage[a4paper, top=2.5cm, bottom=2.5cm, left=2.2cm, right=2.2cm]%
{geometry}
\usepackage{times}
\usepackage{amsmath}
\usepackage{amssymb}
\newenvironment{proof}[1][Proof]{\textbf{#1.} }{\ \rule{0.5em}{0.5em}}

\begin{document}

\title{Math 774 Homework 2}
\author{Edward Kim}
\date{\today}
\maketitle

\subsubsection*{Problem 1 (Humphreys 3.6)}
\begin{proof}
  Let $I,J$ be nilpotent ideals in a Lie algebra $L$. We first note that $[JI] \subseteq I$ and that if we have the following consecutive bracket operations:
  $$ [F_1 [F_2 ... [F_{n-1}[F_n,I]]...]] \subseteq I^{d}, \quad F_i = I,J $$ where $d$ is the number of copies of $I$ contained  in the set of $F_i$. \newline

  We claim that if $I^{k_1} = J^{k_2} = 0$, then $(I + J)^{2(k_1 + k_2)} = 0$. Expanding the product yields terms of length $2(k_1+k_2)$ containing at least $\frac{2(k_1 + k_2)}{2} = k_1 + k_2$ copies of both $I,J$. By our observations above, each term must be contained in either $I^{k_1 + k_2} = 0$ or $J^{k_1 + k_2} = 0$, meaning all terms of $(I+J)^{2(k_1 + k_2)}$ vanish as claimed. \newline

  Further application of the property above shows that there exists a unique maximal nilpotent ideal as having two distinct maximal nilpotent ideals $M_1,M_2$ would yield another nilpotent ideal $M_1 + M_2$ containg both $M_1,M_2$ which would be a contradiction. \newline

  We turn our attention to the three-dimensional Lie algebra $L$ with the following bracket relations:
  $$ [xy] = z, [xz] = y, [yz] = 0 $$ and attempt to first determine the ideal generated by $y$. By taking linear combinations, the following equivalences follow:
  $$ [Fx + Fy + Fz, y] = F[xy] + F[yy] + F[zy] = Fz $$ Furthermore, by taking linear combinations of $z,x$,
  $$ [Fx + Fy + Fz, z] = F[xz] + F[yz] + F[zz] = Fy $$
  $$ [Fx + Fy + Fz, x] = F[xx] + F[yx] + F[zx] = -(Fy + Fz)$$
  The only ideals in $L$ are the subspaces spanned by $y,z$ and $x,y,z$ ($L$ itself). The former is nilpotent as $[Fy+Fz,Fy+Fz] = F^2[yy] + F^2[yz] + F^2[zy] + F^2[zz] = 0$ whereas $L$ is not, making the former our nilradical.

  Lastly, the non-abelian two-dimensional Lie algebra $L$ where $[xy] = x$ only has the zero ideal, the subspace spanned by $x$, and itself as the ideals of $L$:
  By definition, $[Fx,Fx] = 0$, but $L$ does not necessarily have to be since $[LL] \subseteq Fx$. Thus, $Fx$ will be the nilradical.
\end{proof}

\subsection*{Problem 2 (Humphreys 4.1)}
\begin{proof}
 Let $L = \mathfrak{sl}(V)$.
 First, we show that that Rad$(L)$ is contained inside every maximal solvable subalgebra. Suppose that $B$ is a maximal solvable subalgebra such that Rad($L) \not\subseteq B$. Since $B$ is solvable, Rad($B$) $= B$. By taking $N = N_L(\text{Rad}(L) + \text{Rad}(B))$, we yield another solvable ideal in $N$ containing $B$ which would another solvable subalgebra in $L$ containing $B$, contradiction. \newline

 By utilizing Lie's Theorem and choosing a suitable basis of $V$, we can represent all elements in $B$ as upper triangular matrices stabilizing some flag in $V$. Since $B$ is maximal and $V$ finite-dimensional, $B = L \cap \mathfrak{t}(n,F)$ where $n = \text{dim } V$. Let $B^t = \{ b^t \vert b \in B \}$ denote all of the transposed elements of $B$. By taking the transpose of $\mathfrak{t}(n,F)$, we see that $B^t$ consists of lower-triangular matrices. Thus, $B^t$ also stabilizes a flag under a suitable basis change, giving another maximal solvable subalgebra of $L$. By our observations above,
 $$ \text{Rad}(L) \subseteq B \cap B^t = L \cap \mathfrak{d}(n,F) $$
 %We invoke Lie's Theorem again to get an endomorphism $x \in L \cap \mathfrak{d}(n,F)$ such that $x$ is a common eigenvector for all endomorphisms in Rad($L$). Hence,
 %$ \mathfrak{l}x = \lambda(\mathfrak{l})x $ for all $\mathfrak{l} \in \text{Rad }(L)$. However, both $\mathfrak{l}$ and $x$ are trace zero diagonal matrices. So we can rephrase the %above equality in terms of matrix components in $F$ as:
 %$$ \sum_{i=1}^n \mathfrak{l}_{ii}x_{ii} = \lambda(\mathfrak{l}) x_{ii} \implies \sum_{i=1}^n (\mathfrak{l}_{ii} - \lambda(\mathfrak{l}))x_{ii} = 0 $$
 %If we view the $x_{ii}$ as coefficients of basis chosen at the beginning of this proof, then by the linear independence of those basis elements, $\mathfrak{l}_{ii} - %\lambda(\mathfrak{l}) = 0$ and thus $\mathfrak{l}_{ii} = \lambda(\mathfrak{l})$. In light of the fact that $\mathfrak{l}$ is a trace zero diagonal matrix, this
 Thus, Rad($L$) lies in the abelian subalgebra of diagonal trace zero matrices.
 (This is where I get stuck. Diagonal trace zero matrices generally don't commute with every element in $\mathfrak{sl}(V)$).

 By Exercise 2.3 of HW1, Rad$(L)$ = $Z(L)$ = 0. Thus, $L$ is semisimple
\end{proof}

\subsection*{Problem 3 (Humphreys 4.5)}
\begin{proof}
Before the actual proof, we begin with the claim that if $x,y \in \text{End }(V)$ are semi-simple (resp. nilpotent), then the sum $x+y$ is also semi-simple (resp nilpotent). We will use the property that if $x,y$ commute, then they are simultaneously diagonalizable; that is there exists an invertible change of basis matrix $P$ such that $PxP^{-1}$ and $PyP^{-1}$ are both diagonal matrices $D_1,D_2$ respectively. Take $P$ to be the matrix above and computing $P(x+y)P^{-1}$ yields :
$$P(x+y)P^{-1} = PxP^{-1} + PyP^{-1} = D_1 + D_2 $$ which is a diagonal matrix, giving us that the sum is semisimple. The proof for the nilpotentcy of the sum of two nilpotent endomorphisms can be obtained by following the proof outlined in Problem 1: Take the exponent of the sum to be $2(n_1 + n_2)$ where $x^{n_1} = y^{n_2} = 0$. Since $x,y$ commute, each of the terms of the expansion will vanish. \newline
Since $(x_s + y_s) + (x_n + y_n) = (x_s + x_n) + (y_s + y_n) = x + y$ .

By the uniqueness of the Jordan-Chevalley Decomposition, if $x_s + y_s$ and $x_n + y_n$ commute, it must be that $(x + y)_s = x_s + y_s$ and $(x + y)_n = x_n + y_n$.

{\bf Partial Solution For Commutativity:} Expanding $[x_s + y_s, x_n + y_n]$ gives us that:

$$ [x_s + y_s, x_n + y_n] = [x_s,x_n] + [x_s, y_n] + [y_s,x_n] + [y_s,y_n] =
[x_s,y_n] - [x_n, y_s]$$
Since $x,y$ commute
$$[x_s,y_n] + [x_n,y_s] = - ([x_s,y_s] + [x_n,y_n]) $$

This suggests trying to prove that $x_s,y_s$ and $x_n,y_n$ commute, but this seems false.


\end{proof}

\subsection*{Problem 4 (Humphreys 4.7)}
\begin{proof}
  To show the converse, assume that $L$ is a solvable subalgebra in $\mathfrak{gl}(V)$. By Lie's Theorem, $L$ stabilizes a flag in $V$ and thus can be expressed as an element of $\mathfrak{t}(V)$, the upper-triangular matrices of $V$. We recall that
  $\mathfrak{n}(V)$, the set of strictly upper-triangular matrices. is the derived algebra of $\mathfrak{t}(V)$ and that the product of endomorphisms $x \in \mathfrak{n}(V)$ and $y \in \mathfrak{t}(V)$ will be another strictly upper-triangular matrix $xy \in \mathfrak{n}(V)$. Computing the trace on this product gives that $Tr(xy) = 0$ for all $x \in [LL]$ and $y \in L$ as required.
\end{proof}

\subsection*{Problem 5 (Humphreys 5.2)}
\begin{proof}
  First, suppose that the dervied algebra $[LL]$ of Lie algebra $L$ lies in the radical of the Killing form on $L$. Recall that the radical of the Killing form is defined as $S = \{x \in L \vert \kappa(x,y) = 0, \forall y \in L\}$. If $[LL] \subseteq S$, then by definition $Tr$(ad($x$)ad($y$)) = 0 for all $y \in L$ and all $x \in [LL]$. By Cartan's Criterion, $L$ is solvable. \newline
  To show the converse, assume that $L$ is solvable. If so, then ad $(L)$, the adjoint representation, is solvable. We invoke Problem 4 above along with the fact that [ad$(L)$,ad$(L)$] = ad($[LL]$) to get the converse of Corollary 4.3 (Page 20). This directly gives us that $[LL] \subseteq S$.
\end{proof}

\subsection*{Problem 6 (Humphreys 5.3)}
\begin{proof}
  To determine the ad($x$) and ad($y$) matrices, we start with the fact: ad($x$)($y$) = $[xy] = x$. This leaves us with the matrices:
  $$ \text{ad }(x) = \left[ \begin{array}{ccc} 0 & 1 \\ 0 & 0 \end{array} \right], \text{ad }(y) = \left[ \begin{array}{ccc} -1 & 0 \\ 0 & 0 \end{array} \right] $$ where we order the basis as elements as $v_1 = x, v_2 = y$. Computing the trace reveals that:
  $$ Tr( \left[ \begin{array}{ccc} -1 & 0 \\ 0 & 0 \end{array} \right]\left[ \begin{array}{ccc} 0 & 1 \\ 0 & 0 \end{array} \right]) = Tr(\left[ \begin{array}{ccc} 0 & -1 \\ 0 & 0 \end{array} \right]) = 0$$ This shows that the Killing form on the non-abelian two-dimensional Lie-algebra is degenerate.
\end{proof}

\subsection*{Problem 7 (Humphreys 5.5)}
\begin{proof}
  If we order the standard basis elements as $v_1 = x, v_2 = h, v_3 = y$, we claim that the dual basis will be $d_1 = \frac{1}{4}y, d_2 = \frac{1}{8}h, d_3 = \frac{1}{4}x$. This dual basis is evident from following trace calculations:
  \begin{gather}
    Tr(\text{ad }(x)\text{ad }(y)) = Tr(\left[ \begin{array}{ccc} 0 & -2 & 0 \\ 0 & 0 & 1 \\ 0 & 0 & 0  \end{array} \right] \left[ \begin{array}{ccc} 0 & 0 & 0 \\ -1 & 0 & 0 \\ 0 & 2 & 0  \end{array} \right]) = 4 \\
    Tr(\text{ad }(h)\text{ad }(h)) = Tr(\left[ \begin{array}{ccc} 2 & 0 & 0 \\ 0 & 0 & 0 \\ 0 & 0 & -2  \end{array} \right] \left[ \begin{array}{ccc} 2 & 0 & 0 \\ 0 & 0 & 0 \\ 0 & 0 & -2  \end{array} \right]) = 8
  \end{gather}
\end{proof}

\subsection*{Problem 8 (Humphreys 5.6)}
\begin{proof}
  Inspecting the proof of Theorem 5.1 reveals that the only step we have to check in the prime characteristic case is the following statement: Let $x,y \in L$. If ad$x$ad$y$ is nilpotent, then $Tr(\text{ad$x$ }\text{ad}y) = 0$. We use the property of linear algebra that the trace of $x \in$ End$(V)$ is the sum of all eigenvalues of $x$ and the property that if $x$ is nilpotent then its eigenvalues vanish. Both of these properties hold in the prime characteristic case. Combining these properties gives us that $Tr(\text{ad$x$ }\text{ad}y) = 0$ and the proof of the theorem holds. \newline

  As for our counterexample, consider the standard basis matrices $e_{11} - e_{22}$ and $e_{22} - e_{33}$ and their corresponding matrices for their adjoint representations:
  $$\text{ad }(e_{11} - e_{22}) = diag (0,0,2,1,-2,-1,-1,1), \text{ad }(e_{22} - e_{33}) = diag (0,0,-1,1,1,2,-1,-2) $$
  Calculating the Killing Form in respect to these two basis elements yields:
  $$\kappa(e_{11} - e_{22},e_{22} - e_{33}) = Tr(diag (0,0,-2,1,-2,-2,1,-2)) = -6 = 0 $$
  which indicates degeneracy of the Killing Form on $\mathfrak{sl}(3,F)$.
\end{proof}


\subsection*{Problem 9 (Humphreys 6.2)}
\begin{proof}
  We first assume that the $L$-module $V$ can be decomposed into a direct sum of irreducible $L$-submodules $\bigoplus_i V_i$. Given any submodule $S \subseteq V$, $S$ can also be decomposed into direct sum of irreducible submodules which must be a subsum of $\bigoplus_i V_i$ (if $S \cap V_i \neq 0$, then either $V_i \subseteq S$ or $S  = V_i$. If the latter, then we are done. Otherwise, continue with the quotient module $S/V_i$). It follows that $V/S$ will be the complement of $S$ as the sum is direct. Conversely, assume that every submodule of $V$ has a complement in $V$. We outline a procedure to output a direct sum of irreducible submodules: If $V$ is irreducible already, then the procedure halts. Else take a minimal non-trivial submodule $V_1$. By definition, $V_1$ must be irreducible and $V_1$ has a complement $X_1$ such that $V = V_1 \oplus X_1$. Induct on $X_1$. This procedure must terminate as the $V_i$ are non-trivial and $V$ is of finite dimension. The resulting direct sum is the desired sum of irreducible submodules.
\end{proof}

\subsection*{Problem 10 (Humphreys 6.3)}
\begin{proof}
  Let $\phi: L \rightarrow \mathfrak{gl}(V)$ be an irreducible representation of the solvable Lie algebra $L$. Since $\phi(L)$ is also solvable, we invoke Lie's Theorem on $\phi(L)$ to deduce that the endomorphisms representing the elements of $L$ must stabilize some flag. Hence, they must be in upper-triangular form for a properly chosen basis of $V$. Recall that a flag is chain of ascending subspaces : $0 = V_0 \subset V_1 \subset ... \subset V_n = V $ where dim($V_i$) = $i$. Since $\phi(L)$ stabilizes this flag, the subspaces are actually $L$-submodules. By irreducibility of $V$, the flag must be of the form $ 0 = V_0 \subset V_1 = V$, showing that $V$ is one-dimensional as desired.
\end{proof}

\subsection*{Problem 11 (Humphreys 6.5)}
\begin{proof}
  \begin{enumerate}
    \item If $L$ is reductive, then there is an isomorphism $\phi: L / \text{Rad}(L) \rightarrow \text{ad }(L)$ which in particular is a representation $\phi:L / \text{Rad}(L) \rightarrow \mathfrak{gl}(L)$. As $L / \text{Rad}(L)$
    is semi-simple, Weyl's Theorem shows us that $L$ is completely reducible as ad($L$)-module. Furthermore, $[LL] \cong [L/Z(L),L/Z(L)] \cong L/Z(L)$. The last isomorphism follows from the semi-simplicity of $L/Z(L)$ (See Corollary 5.2). Combining the first assertion with the observation that $Z(L)$ is an irreducible ad $(L)$-submodule (ad acts trivially on $Z(L)$), we deduce that $L = L/Z(L) \oplus Z(L)$ and thus $L = [LL] \oplus Z(L)$.

    \item By Problem 2, we know that $\mathfrak{sl}(n,F)$ is semisimple. Since types $\mathfrak{sp}(2n,F),\mathfrak{o}(2n+1,F),\mathfrak{o}(2n,F)$ all consist of trace zero matrices, they must also be semi-simple as well (as any solvable ideal in these subalgebras would be solvable in $\mathfrak{sl}(n,F)$).
    \item Suppose that $L$ is a completely reducible ad $(L)$-module such that $L = \bigoplus_i L_i$. We first note that Rad $(L)$ and $Z(L)$ are both ad $(L)$- submodules, In particular, $Z(L)$ is an irreducible submodule. By the hypothesis, we can decompose $L = Z(L) \oplus X$ where $X$ is another direct sum of irreducible submodules. Take the quotient $L/Z(L)$. It must be semi-simple as it has a decomposition as a direct sum of simple ideals (which are irreducible ad $L$-submodules). By one of our isomorphism theorems, Rad$(L)$/$Z(L)$ is an ideal of $L/Z(L)$, showing that it must also be semi-simple. However, if Rad$(L)$/$Z(L)$ is semisimple, then there exists no solvable ideals containing $Z(L)$ in Rad$(L)$, giving that Rad($L$) = $Z(L)$ and that $L$ is reductive.

    \item From Part 1 of this problem, $L = [LL] \oplus Z(L)$ where $[LL]$ is semisimple. Let $\phi: L \rightarrow \mathfrak{gl}(V)$ be a representation where the center acts by semisimple endomorphisms. We can view $\phi: [LL] \oplus Z(L) \rightarrow \mathfrak{gl}(V)$. Since $\phi(Z(L))$ are semisimple and because they commute with each other, they are simultaneously diagonalizable according to suitable basis say $b_1,...,b_n$ where $n = \text{dim }(V)$. Thus, the $\phi(Z(L))$ act on the $b_i$ by scaling by the respective eigenvalues given by the linear map $\lambda_i:\phi(Z(L)) \rightarrow F$, stabilizing each subspace spanned by the individual basis elements. Define an $[LL]$-module structure on $V$ by taking $l \in [LL]$ to the endomorphism $\phi(l)\cdot v$ where $v$ is expressed as a linear combination of the $b_i$. By the semisimplicity of $[LL]$ and Weyl's Theorem, $V$ must be completely reducible.
   \end{enumerate}
   (Casual Note: I wanted to say that the $[LL]$-module above is somehow equivalent to a representation from $L/Z(L) \rightarrow \mathfrak{gl}(V)$. I felt like I had to quotient $\mathfrak{gl}(V)$ with something like $\mathfrak{d}(V)$ since I wanted to quotient out all the scaling by eigenvalues induced by $Z(L)$ actions.
 \end{proof}

\subsection*{Problem 12 (Humphreys 6.6)}
\begin{proof}
 Our general idea will be to try to find an endomorphism in $\mathfrak{gl}(L)$ involving the two forms $\beta,\gamma$ which commutes with ad($L$). Schur's lemma will tell us that this endomorphism will be a scalar in $F$. In addition, let $z \in L$ and suppose $x,y \in L$ such that $\beta(x,z) = \gamma(y,z)$. If we can prove that $y = \lambda x$, then by bilinearity of the forms, we will be finished. This spurs some motivation to define a map which takes $\beta$ and finds a value say $y_{\beta}$ such that $\beta(x,z) = \gamma(y_{\beta},z)$ for all $z \in L$. We could define such a map as follows: let $g: L^* \rightarrow L$ be defined by taking $r \mapsto y_r$ where $r(z) = \gamma(y_r,z)$ for all $z \in L$. This then gives us a natural definition for using $\beta$ by defining $f: L \rightarrow L^*$ by taking $x \mapsto \beta(x,-)$. Composing these two functions maps every $x \in L$ to $y_{\beta(x,-)} \in L$. We must now show that $gf \in \text{End}(L)$ commutes with ad ($L$) as follows: Define an $\text{ad}L$ action on $L^*$ as follows: $(\text{ad}x \cdot \phi)(z) = \phi([zx]) = -\phi([xz])$. Now we shall prove that maps $f,g$ are both ad $L$-homomorphisms:

 $$ f(\text{ad}x(y))(z) = (f([x,y]))(z) = \beta([xy],z) = -\beta([yx],z) = -\beta(y,[xz]) = ((\text{ad}x) \cdot f(y))(z) $$
 We do the same to $g$ by expanding the definitions:
 $$ \phi([xz]) = \gamma(y_{\phi},[xz]) = - \gamma([xy_{\phi}],z) = \text{ad}x \cdot g(\phi) $$ Thus,
 $ g(\text{ad}x \cdot \phi)(z) = (\text{ad }x) \cdot g(\phi)$.
 By chaining the two properties above, we get that $gf$ commutes with ad$L$:
 $$(g \circ f)(\text{ad}x)(z) = g(f(\text{ad }x(z))) = g((\text{ad}x \cdot f)(z)) =  (\text{ad}x \cdot (g \circ f))(z)$$
  By our explanation above, this means $gf = \lambda$ where $\lambda$ is a scalar. So $gf(x) = y_{\beta(x,-)} = \lambda x$ as desired.
\end{proof}

\subsection*{Problem 13 (Humphreys 6.7)}
\begin{proof}
By Exercise 6, we know that $\kappa(x,y) = \lambda Tr(x,y)$ for all $x,y \in \mathfrak{sl}(n,F)$. It suffices to compute this for one pair of values in $\mathfrak{sl}(n,F)$. We start by computing the eigenvalues for ad($e_{22} - e_{33}$) using the standard basis matrices. From this we get that:
 \begin{enumerate}
   \item The $e_{ii} - e_{i+1,i+1}$ have eigenvalue 0
   \item $e_{3,2}$ and $e_{2,3}$ have eigenvalues -2,2 respectively
   \item $e_{n2}$ and $e_{3n}$ for $n \neq 2,3$ have eigenvalues -1
   \item $e_{2n}$ and $e_{n3}$ for $n \neq 2,3$ have eigenvalues 1.
   \item All other $e_{i,j}, i \neq j$ have eigenvalue 0
 \end{enumerate}
 By fixing our basis order to have the eigenvalue 0 basis elements first, we create the following matrix for ad$(e_{22} - e_{33})$
 $$ \text{ad}(e_{22} - e_{33}) = \text{diag }(0,....,0,2,-2,-1,....,-1,1,.....,1) $$
 Taking the trace of the square of this matrix yields:
 $$Tr(\text{ad}(e_{22} - e_{33})) = 8 + 2(2(n-2)) = 4n$$
 Now since $Tr(e_{22} - e_{33}) = 2$, it follows that $\lambda = 2n$ as required.
\end{proof}

\end{document}
