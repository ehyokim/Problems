\documentclass[12pt]{article}%
\usepackage{amsfonts}
\usepackage{amsmath}
\usepackage{amssymb}

\usepackage{geometry}
\usepackage{physics}

\usepackage{times}
\usepackage{microtype}


\usepackage{mathtools}
\usepackage{hyperref} % backref=page
\usepackage[shortlabels]{enumitem}

% Prints a trailing space in a smart way.
\usepackage{xspace}
\usepackage[font=footnotesize]{caption}
\usepackage{titlesec}

\DeclarePairedDelimiter\ceil{\lceil}{\rceil}
\DeclarePairedDelimiter\floor{\lfloor}{\rfloor}
\newenvironment{proof}[1][Proof]{\textbf{#1.} }{\ \rule{0.5em}{0.5em}}


% General Mathematical Notation
\newcommand{\reals}{{\mathbb{R}}}
\newcommand{\complex}{{\mathbb{C}}}                    %reals                  %reals
\newcommand{\nnreal}{{\nnnum R}}
                   %nonnegative

\newcommand{\nat}{\mathbb{N}}                      %natural numbers
\newcommand{\integers}{{\num Z}}                      %integers
\newcommand{\rat}{{\num Q}}                      %rationals
\newcommand{\nnrat}{{\nnnum Q}}

\newcommand{\boldu}{\textbf{u}}
\newcommand{\boldv}{\textbf{v}}
%\newcommand{\ip}[1][2]{\langle {#1}, {#2},\rangle}

% Differentiation
\newcommand{\totder}[1][]{\frac{d^{#1}}{dx^{#1}}}

% Environments
\newtheorem{theorem}{Theorem}[section]
\newtheorem{lemma}{Lemma}[section]

% Series
\newcommand{\powaseries}[1][0]{\sum_{n = {#1}}^{\infty}}

% Vector Spaces

% Hilbert Space
\newcommand{\hil}{\mathcal{H}}
\newcommand{\bhilop}{\mathcal{B}(\mathcal{H})}


\geometry{letterpaper,left=1in,top=1in,right=1in,bottom=1in, footskip=0.5in, nohead}


\newcommand{\Uone}{\mathcal{U}_1}
\newcommand{\Utwo}{\mathcal{U}_2}
\newcommand{\Uint}{\Uone \cap \Utwo}
\newcommand{\V}{\mathcal{V}}
\newcommand{\gpart}[1][t]{\frac{\partial g(x,t)}{\partial {#1}}}
\setlength\parindent{0pt}
\newcommand{\Legen}[1][n]{(x^2 - 1)^{#1}}

\newcommand{\normphi}[1][v]{\abs{\Phi\left({#1}
\right)}}

\newcommand{\disip}[2]{\left\langle #1, #2 \right\rangle}
\newcommand{\infsum}[1][0]{\sum^{\infty}_{n = {#1}}}
\newcommand{\posinfint}[1][0]{\int_{#1}^\infty}
\newcommand{\neginfint}[1][0]{\int_{-\infty}^{#1}}

% Fourier Shortcuts
\newcommand{\piinter}{[-\pi,\pi]}
\newcommand{\pifrac}{\frac{1}{\pi}}
\newcommand{\piint}[2][x]{\int_{-\pi}^\pi {#2} \; d{#1}}
\newcommand{\twopiint}[2][x]{\int_{0}^{2\pi} {#2} \; d{#1}}
\newcommand{\foutrans}[1][k]{\tilde{f}({#1})}

% Misc
\newcommand{\parafrac}[2]{\para{\frac{#1}{#2}}}

\title{PHY510 HW 9}
\author{Edward Kim}
\date{\today}

\begin{document}
\maketitle

\section*{Problem 1}

\begin{enumerate}
    \item We will use the trick introduced in Stone and Goldbart for this portion. As detailed, the function we wish to calculate is:

    \begin{align*}
        f(z) = 2u\left(\frac{z}{2},\frac{z}{2i} \right) & = 2\left[\para{\frac{z}{2}}^4 - 6\para{\frac{z}{2}}^2\para{\frac{z}{2i}}^2 + \para{\frac{z}{21}}^4\right] \\
        & = 2 \left(\frac{z^4}{16} + 6 \frac{z^4}{16} + \frac{z^4}{16}\right) \\
        & = z^4
    \end{align*}

    This will be our analytic function up to some constant $iC$ for some $C \in \reals$. As a quick sanity check, computing the real and imaginary functions $u(x,y), v(x,y)$ reveals that:

    \[ u(x,y) = x^4  - 6xy + y^4, \quad v(x,y) = 4x^3y - 4xy^3\]

    \item It appears simpler to use the trick in this case. As above, we tailor the trick to the case when we are given the imaginary component $v(x,y)$:

    \def \PartTwoNumerA {2 \parafrac{z}{2} \parafrac{z}{2i}}
    \def \PartTwoDenomA { \parafrac{z}{4}^4 + \parafrac{z}{2i}^4 + 2 \parafrac{z}{2}^2 \parafrac{z}{2i}^2 + 2 \parafrac{z}{2}^2 - 2 \parafrac{z}{2i}^2 + 1 }

    \def \PartTwoDenomB {\cancel{\parafrac{z^4}{16}} + \cancel{\parafrac{z^4}{16}} - \cancel{2 \parafrac{z^4}{16}} + 2\parafrac{z^2}{4} + 2\parafrac{z^2}{4} + 1}

    \begin{align*}
      f(z) = 2i \cdot v\left(\frac{z}{2}, \frac{z}{2i} \right) & = -2i \left[\frac{\PartTwoNumerA}{\PartTwoDenomA} \right]\\
      & = \frac{-z^2}{\PartTwoDenomB} \\
      & = - \frac{z^2}{z^2 + 1}
    \end{align*}

    This will be our analytic function up to some real constant $C \in \reals$.

    \item We solve this portion using direct integration. We first calculate:

    \[ \partder{v}{x} = e^{-y}\cos(x) \quad \partder{v}{y} = -e^{-y}\sin(x) \]

    By the Cauchy-Riemann Equations, we can integrate over the proper partial derivatives to find $v(x,y)$:

    \begin{align*}
      u(x,y) & = \int \partder{u}{x} dx = \int -e^{-y}\sin(x) dx = e^{-y}\cos(x) + \phi(y) \\
      u(x,y) & = \int \partder{u}{y} dy = - \int e^{-y}\cos(x) dy = e^{-y}\cos(x) + \psi(x)
    \end{align*}

    where $\psi(x), \phi(y)$ are both terms depending only on $x,y$ respectively. However, inspecting the results shows that $\phi(y) = \psi(x) = C\in \reals$. Equating these two expressions together,

    \[ u(x,y) =  e^{-y}\cos{x} + C, \; v(x,y) = e^{-y}\sin{x}\]

    In the case where $C=0$, the analytic function which satisfies these constraints must be

    \[ w(z) = e^{iz} \]
\end{enumerate}

\section{Problem 2a ([H] 10.5)}

We start with the fact that if $f$ is analytic, then its complex derivative can be expressed as \[\totder{f}{z} = \partder{u}{x} + i\partder{v}{x}\]
%
By the chain rule, we derive that:

\begin{align}
  \partder{u}{r} & = \partder{u}{x} \cos(\theta) + \partder{u}{y}\sin(\theta) \label{eq:urRel}\\
  \partder{v}{r} & = \partder{v}{x} \cos(\theta) + \partder{v}{y}\sin(\theta) \label{eq:uvRel}
\end{align}

Subsequently, we directly calculate the following:
\begin{align*}
  \partder{u}{r} + i\partder{v}{r} & = \left[\partder{u}{x} \cos(\theta) + \partder{u}{y}\sin(\theta)\right] + i\left[\partder{v}{x} \cos(\theta) + \partder{v}{y}\sin(\theta)\right] \\
  & = \left[\partder{u}{x} \cos(\theta) - \partder{v}{x}\sin(\theta)\right] + i\left[\partder{v}{x} \cos(\theta) + \partder{u}{x}\sin(\theta)\right] \\
  & = \partder{u}{x}\left[\cos(\theta) + i\sin(\theta)\right] + \partder{v}{x}\left[-\sin(\theta) + i\cos(\theta)\right] \\
  & = \partder{u}{x}\left[\cos(\theta) + i\sin(\theta)\right] + i \partder{v}{x}\left[\cos(\theta) + i\sin(\theta)\right] \\
  & = e^{i\theta}\left[\partder{u}{x} + i\partder{v}{x}\right]
\end{align*}
where we get the second equality by invoking the Cauchy-Riemann Conditions.

Multiply both sides of the equation by $e^{-i\theta}$ and invoke our first comment to yield the desired form \[\totder{f}{z}= e^{-i\theta} \left[\partder{u}{r} + i\partder{v}{r} \right]\]

Additionally, consider the equalities derived once more from the chain rule:

\begin{align}
  \partder{u}{\theta} & = - \partder{u}{x}\left(r \sin(\theta)\right) + \partder{u}{y}\left(r\cos(\theta)\right) \label{eq:uthetaRel} \\
    \partder{v}{\theta} & = - \partder{v}{x}\left(r \sin(\theta)\right) + \partder{v}{y}\left(r\cos(\theta)\right) \label{eq:vthetaRel}
\end{align}

Now we combine Equations \ref{eq:urRel}, \ref{eq:vthetaRel}:

\begin{align*}
  \partder{u}{r} - \frac{1}{r}\partder{v}{\theta} & = \left[\partder{u}{x} \cos(\theta) + \partder{u}{y}\sin(\theta)\right] - \frac{1}{r}\left[- \partder{v}{x}\left(r \sin(\theta)\right) + \partder{v}{y}\left(r\cos(\theta)\right) \right] \\
  & = \cancel{\left[\partder{u}{x} - \partder{v}{y}\right]} \cos(\theta) + \cancel{\left[\partder{u}{y} +  \partder{v}{x}\right]} \sin(\theta) \\
  & = 0
\end{align*}

The factors cancel due to the CR conditions. Additionally, we can apply the same idea to Equations \ref{eq:uvRel},\ref{eq:uthetaRel}:

\begin{align*}
  \partder{v}{r} + \frac{1}{r}\partder{u}{\theta} & = \left[\partder{v}{x} \cos(\theta) + \partder{v}{y}\sin(\theta)\right] + \frac{1}{r}\left[- \partder{u}{x}\left(r \sin(\theta)\right) + \partder{u}{y}\left(r\cos(\theta)\right) \right] \\
  & = \cancel{\left[\partder{v}{y} - \partder{u}{x}\right]}\sin(\theta) + \cancel{\left[\partder{v}{x} + \partder{u}{y}\right]}\cos(\theta) \\
  & = 0
\end{align*}
Hence, the Cauchy-Riemann Conditions expressed in polar coordinates turn out to be \[\partder{u}{r} =  \frac{1}{r}\partder{v}{\theta}, \quad \partder{v}{r} = - \frac{1}{r}\partder{u}{\theta}\]

\section{Problem 2b [H] 10.5}

Using the definitions of $\sin(x), \cos(x), \sinh(y), \cosh(y)$, the equalities below must hold:

\begin{align*}
  \sin(z) & = \sin(x + iy) \\
          & = \cos(iy)\sin(x) + \sin(iy)\cos(x) \\
          & = \left[\frac{e^{-y} + e^{y}}{2}\right]\sin(x) + \left[\frac{e^{-y} - e^{y}}{2i}\right]\cos(x) \\
          & = \cosh(y)\sin(x)+ i \sinh(y)\cos(x)
\end{align*}
where the second equality follows from the standard sine angle addition formula and
\begin{align*}
  \cos(z) = \cos(x + iy) & = \cos(x)\cos(iy) - \sin(x)\sin(iy) \\
  & = \cos(x)\left[\frac{e^{-y} + e^{y}}{2}\right] - \sin(x)\left[\frac{e^{-y} - e^{y}}{2i}\right] \\
  & = \cos(x)\cosh(y) - i \sin(x)\sinh(y)
\end{align*}

Now consider the roots of $\sin(z)$ first. This imposes the constraints:

\begin{align*}
  \cosh(y)\sin(x) = 0 \\
  \sinh(y)\cos(x) = 0
\end{align*}
Observe that $\cosh(y) \neq 0$ for all $y \in \reals$ as both $e^y, e^{-y} > 0$. This would force $\sin(x) = 0$. However $\cos(x),\sin(x)$ do not share any roots in common. Thus, this would further force $\sinh(y) = 0$. This can only happen if $y = 0$ i.e $z \in \reals$. \newline

Similarly, the two constraints imposed for $\cos(z) = 0$ will be
\begin{align*}
  \cos(x)\cosh(y) =0 \\
   \sin(x)\sinh(y) = 0
\end{align*}

Following the same line of logic $\cos(x) = 0$ since $\cosh(y) \neq 0$. So $\sinh(y) = 0$ iff $y =0$.

\section*{Problem 2c [H] 10.12}

\begin{enumerate}[i.]
  \item $\mathfrak{Re}\sin(z) = \sin(x)\cosh(y)$ by Problem 2b above.
  \item
  We calculate that:
  \begin{align*}
    \cosh(z) & = \frac{e^{x+iy} + e^{-(x+iy)}}{2} \\
    & = \frac{1}{2}\left[e^x(\cos(y) + i\sin(y)) + e^{-x}(\cos(y) - i\sin(y))\right] \\
    & = \frac{1}{2}\left[(e^x + e^{-x})\cos(y) + i(e^x - e^{-x})\sin(y)\right] \\
    & = \cosh(x)\cos(y) + i\sinh(x)\sin(y)
  \end{align*}
  Thus, $\mathfrak{Im}\cosh(z) = \sinh(x)\sin(y)$

  \item
  Proceeding as before with the calculation shown in Problem 2b:
  \begin{align*}
    \abs{\sin(z)}^2 & = \abs{\cosh(y)\sin(x)+ i \sinh(y)\cos(x)}^2 \\
    & = \cosh^2(y)\sin^2(x) + \sinh^2(y)\cos^2(x) \\
    & = \left[\frac{e^{2y} + e^{-2y} + 2}{4}\right]\sin^2(x) + \left[\frac{e^{2y} + e^{-2y} - 2}{4}\right]\cos^2(x) \\
    & =  \frac{e^{2y} + e^{-2y}}{4} + \frac{1}{2} (\sin^2(x) - \cos^2(x)) \\
    & = \left[ \frac{e^{2y} + e^{-2y} - 2}{4}\right] + \frac{1}{2}(\sin^2(x) - \cos^2(x) + 1) \\
    & = \sinh^2(y) + \sin^2(x)
  \end{align*}
  where in the second-to-last equality we added a vanishing term of $- \frac{1}{2} + \frac{1}{2}$

  \item
  This case follows similarly to the third case above:
  \begin{align*}
    \abs{\cosh(z)}^2 & = \cosh^2(x)\cos^2(y) + \sinh^2(x)\sin^2(y) \\
    & = \left[\frac{e^{2y} + e^{-2y} + 2}{4}\right]\cos^2(y) + \left[\frac{e^{2y} + e^{-2y} - 2}{4}\right]\sin^2(y) \\
    & = \left[\frac{e^{2x} + e^{-2x}}{4}\right] + \frac{1}{2}(\cos^2(y) - \sin^2(y)) \\
    & = \left[\frac{e^{2x} + e^{-2x} - 2}{4}\right] + \frac{1}{2}(\cos^2(y) - \sin^2(y) + 1) \\
    & =\sinh^2(x) + \cos^2(y)
  \end{align*}
\end{enumerate}

\section*{Problem 2d (b)}

Expandin the complex exponential and equating the components shows that:

\begin{align*}
  e^x\cos(y) & = 1 \\
  e^x\sin(y) & = \sqrt{3}
\end{align*}

If we square both equations and add the results, we see that $e^{2x} = 4$. This forces $x = \ln{2}$. Furthermore, since $e^x > 0$. Both $\cos(y), \sin(y) > 0$, meaning $y$ must lie on the first quadrant of the unit circle. In fact, the only points which satisfy the constraints above will be $\frac{\pi}{3} + 2n\pi$ for $n \in \mathbb{Z}$. Thus, the complex numbers which satisfy the equation above will be \[ \ln{2} + i\left(\frac{\pi}{3} + 2n\pi \right) \quad n \in \mathbb{Z} \]

\section*{Problem 3}
\begin{enumerate}
  \item We show this directly. Define $z' = \frac{az + b}{cz + d}$ and $z'' = \frac{\alpha z + \beta}{\gamma z + \delta}$.
  \begin{align*}
      z''(z) & = \frac{\alpha \frac{az + b}{cz + d} + \beta}{\gamma \frac{az + b}{cz + d} + \delta} \\
      & = \frac{\alpha(az + b) + \beta(cz+d)}{\gamma(az+b) + \delta(cz+d)} \\
      & = \frac{(\alpha a + \beta c)z + (\alpha b + \beta d)}{(\gamma a + \delta c)z + (\gamma b  + \delta d)}
  \end{align*}
  In accordance with:
  \begin{equation*}
    \begin{bmatrix}
      \alpha & \beta \\
      \gamma & \delta
    \end{bmatrix}  \begin{bmatrix}
        a & b\\
        c & d
      \end{bmatrix} = \begin{bmatrix}
        \alpha a + \beta c  & \alpha b + \beta d \\
         \gamma a + \delta c   & \gamma b + \delta d
    \end{bmatrix}
  \end{equation*}
  as expected.

  \item
  Consider the claimed cross ratio identity for M\"{o}bius transformation $w$:
  \begin{equation*}
    \frac{(w - w_3)(w_2-w_4)}{(w - w_4)(w_2 - w_3)} =   \frac{(z - z_3)(z_2-z_4)}{(z - z_4)(z_2 - z_3)}
  \end{equation*}

  Observe by intepreting both sides as functions of $z$, the two expressions align at points $z_2,z_3,z_4$, namely they equal $1,0,\infty$ respectively.

  Multiply the left side by $\frac{cz+d}{cz+d}$ to cancel the denominators of both $w$ and expand out the fraction by cross multiplcation. By moving all the terms to the left side, we see that this yields a polynomial in $z$ of degree at most two. However, we know that this polynomial must vanish at least three points $z_1,z_2,z_3$. Hence, the identity must hold over all $z$ as this polynomial must be zero everywhere.

  Now by solving for $w$ will yield an expression in terms of $z_2,z_3,z_4,w_2,w_3,w_4$ and independent variable $z$. The values $w_i$ are completely determined by $w$.

  \item
  Consider the map:
  \begin{align*}
      w(z) = - \frac{(z -z_1)(z_2-z_3)}{(z - z_3)(z_1 - z_2)}
  \end{align*}

  This map takes $z_1 \mapsto 0$, $z_2 \mapsto 1$, $z_3 \mapsto \infty$ as required. This can be found through direct inspection or the result from the previous part. We shall show the latter. As mentioned before, we wish to find a M\"{o}bius transformation which maps $z_1 \mapsto 0$, $z_2 \mapsto 1$, $z_3 \mapsto \infty$. Invoke the identity in part two to see that

  \begin{align*}
      1 - w & = \frac{(z-z_2)(z_1-z_3)}{(z-z_3)(z_1-z_2)} \\
       w  & = -  \frac{(z-z_2)(z_1-z_3)}{(z-z_3)(z_1-z_2)} + 1 \\
      & = \frac{((z_1 - z_2) - (z_1 - z_3))z + (z_2(z_1 - z_3) - z_3(z_1 - z_2))}{(z-z_3)(z_1-z_2)} \\
      & =  - \frac{(z -z_1)(z_2-z_3)}{(z - z_3)(z_1 - z_2)}
  \end{align*}



  %Observe that the identity of part two asserts that the cross ratio is invariant under M\"{o}bious transformations.


  To check that $w$ is indeed a M\"{o}bius transformation, it suffices to check that $ad - bc \neq 0$. This is equivalent to the $ad - bc = 1$ as the inverse of the square root of the determinant can be multiplied to the numerator and denominator of $w(z)$. Note that this does not change the M\"{o}bius transformation but does scale the determinant to unity. To this end

  \begin{align*}
    \det \begin{bmatrix}
        -(z_2 - z_3) & z_1(z_2 - z_3) \\
        z_1 - z_2 & -z_3(z_1 - z_2)
  \end{bmatrix} & = z_3(z_1 - z_2)(z_2 - z_3) - z_1(z_2 - z_3)(z_1 - z_2) \\
  & = (z_3 - z_1)(z_2 - z_3)(z_1 - z_2)
  \end{align*}
  If the points $z_1,z_2,z_3$ are all distinct, then this quantity will be non-vanishing.
\end{enumerate}

\end{document}
